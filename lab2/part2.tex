\subsection{Введение}

Чтобы понять, при каком значении сопротивления в цепи будет наблюдаться \textbf{колебательный} процесс, а при каком \textbf{апериодический}, нужно найти характеристическое сопротивление, которое для цепи второго порядка определяется как:

\[
	p = \sqrt{\frac{L}{C}} = \sqrt{\frac{0,48}{300 \cdot 10^{-6}}} = 40 \, \text{Ом}
\]

На основе этой величины можно понять какой будет процесс: Если $R > p$, то затухания колебаний ускоряются и энергия системы не возвращается в исходное состояние, если $R = 2 \cdot p$ -- это критический момент, когда процесс перестаёт быть апериодическим, но возвращается в равновесное состояние без колебаний, а при $R < 2 \cdot p$ энергия будет поочерёдно передаваться между индуктивностью и ёмкостью, вызывая затухающие колебания.

Таким образом, при $R = 4 \cdot p = 4 \cdot 40 = 160 \, \text{Ом}$ наблюдается \textit{апериодический} процесс, а при $R = \frac{p}{2} = \frac{40}{2} = 20 \, \text{Ом}$ -- \textit{колебательный}.


% -------------------------------


\subsection{Исследование апериодического процесса}

\subsubsection{Схема исследуемой цепи}
На рисунке 3 представлена схема замещения цепи второго порядка, состоящей из генератора прямоугольного напряжения с резистивной, ёмкостной и индуктивной нагрузками, созданная в приложении LTspice.

\begin{figure}[H]
	\centering
	\includegraphics[width=1\textwidth]{./data/rcl_1-schema.png}
	\caption{Схема замещения RCL-цепи в LTspice.}
\end{figure}

\subsubsection{Расчётные формулы и расчёты}
\begin{enumerate}
	\item Применим типивые эквивалентные преобразования к \textbf{исходной схеме}:
	      \newcommand{\given}[1]{%
	\begin{figure}[H]
		\centering
		\begin{circuitikz}[american, scale=#1]

			\draw
			(0,0)
			to[R, l=$R_3$] (10,0)
			-- (10, -3)
			to[short, *-] (10, -4.1);

			\draw
			(10, -3)
			to[R, l=$R_5$] (7.5, -3)
			to[I, l=$E_5$] (5, -3)
			to[short, *-] (5, -3)
			to[R, l=$R_2$] (2.5, -3)
			to[I, l=$E_2$] (0, -3);

			\draw
			(0,0)
			-- (0,-3)
			to[short, *-] (0, -3)
			to[R, l=$R_1$] (0, -6)
			-- (5, -6) to[short, *-] (10, -6)
			-- (10, -4.9);

			\draw
			(5, -3)
			to[R, l=$R_4$] (5, -6);


			\draw
			(10,-4.5) circle(0.4cm)
			(10,-4.45) edge[thick, -{Straight Barb[length=2mm]}] (10, -4.44)
			(10,-4.35) edge[thick, -{Straight Barb[length=2mm]}] (10, -4.34)
			node at (10.75, -4.5) {$J_6$};

		\end{circuitikz}
		\caption{Исходная электрическая схема с обозначениями элементов}
	\end{figure}
}

	      \given{1}
	\item Расщепляем $J_6$ на $R_4, R_5, E_5$:
	      \begin{figure}[H]
	\centering
	\begin{circuitikz}[american, scale=1]

		\draw
		(0,0)
		to[R, l=$R_3$] (10,0)
		-- (10, -2)
		to[short, *-] (10, -2);

		\draw
		(10, -2)
		to[R, l_=$R_5$] (7, -2)
		to[I, l=$E_5$] (5, -2)
		to[short, *-] (5, -2)
		to[R, l=$R_2$] (2.5, -2)
		to[I, l=$E_2$] (0, -2);

		\draw
		(0,0)
		-- (0,-2)
		to[short, *-] (0, -3)
		to[R, l=$R_1$] (0, -6)
		-- (5, -6) to[short, *-] (5, -6);

		\draw
		(5, -2)
		-- (5, -3)
		to[R, l_=$R_4$] (5, -6);


		\draw
		(9.5, -2)
		to[short, *-] (9.5, -2)
		-- (9.5, -3)
		-- (8.85, -3);

		\draw
		(7.5, -2)
		to[short, *-] (7.5, -2)
		-- (7.5, -3)
		-- (8.05, -3);

		\draw[rotate around={-90:(8.5,-3)}]
		(8.5,-3.05) circle(0.4cm)
		(8.5,-2.95) edge[thick, -{Straight Barb[length=2mm]}] (8.5, -2.94)
		(8.5,-2.85) edge[thick, -{Straight Barb[length=2mm]}] (8.5, -2.84)
		node at (9.25, -3) {$J_{\text{э2}}$};

		\draw
		(5, -3.5)
		to[short, *-] (5, -3.5)
		-- (6, -3.5)
		-- (6, -4.1);

		\draw
		(5, -5.5)
		to[short, *-] (5, -5.5)
		-- (6, -5.5)
		-- (6, -4.9);

		\draw
		(6,-4.5) circle(0.4cm)
		(6,-4.45) edge[thick, -{Straight Barb[length=2mm]}] (6, -4.44)
		(6,-4.35) edge[thick, -{Straight Barb[length=2mm]}] (6, -4.34)
		node at (6.75, -4.5) {$J_{\text{э1}}$};


	\end{circuitikz}
\end{figure}

	\item $J_{\text{э}1}$ \parallel $R_4 \rightarrow E_4$, $J_{\text{э}2}$ \parallel $R_5 \rightarrow E_{5'}$:
	      \begin{figure}[H]
	\centering
	\begin{circuitikz}[american, scale=1]

		\draw
		(0,0)
		to[R, l=$R_3$] (10,0)
		-- (10, -2)
		to[short, *-] (10, -2);

		\draw
		(10, -2)
		to[R, l=$R_5$] (8, -2);

		\draw
		(7, -2) to[I, l_=$E_{5'}$] (8, -2);

		\draw
		(7, -2)
		to[I, l=$E_5$] (5, -2)
		to[short, *-] (5, -2)
		to[R, l=$R_2$] (2.5, -2)
		to[I, l=$E_2$] (0, -2);

		\draw
		(0,0)
		-- (0,-2)
		to[short, *-] (0, -5)
		to[R, l=$R_{14}$] (3, -5)
		to[I, l=$E_4$] (5, -5)
		-- (5, -2);
	\end{circuitikz}
\end{figure}


	      \[
		      \begin{gathered}
			      E_4 = R_4 \cdot J_{\text{э1}} = 2 \cdot 1.95 = 3.9 \, \text{В} \\
			      E_{5'} = R_5 \cdot J_{\text{э2}} = 7 \cdot 1.95 = 13.65 \, \text{В} \\
			      R_{35} = R_3 + R_5 = 14 \, \Omega \\
			      E_{55'} = E_5 - E_{5'} = 6.65 \, \text{В}
		      \end{gathered}
	      \]
	\item $R_{14}, E_4$ \parallel $R_{35}, E_{55'}$:
	      \begin{figure}[H]
	\centering
	\begin{circuitikz}[american, scale=1]

		\draw
		(0,0)
		-- (10, 0)
		-- (10, -2)
		to[R, l=$R_{1435}$] (8, -2)
		to[I, l=$E_{455'}$] (5, -2)
		to[R, l=$R_2$] (2.5, -2)
		to[I, l=$E_2$] (0, -2)
		-- (0, 0);

	\end{circuitikz}
\end{figure}


	      \[
		      \begin{gathered}
			      R_{1435} = \frac{1}{\frac{1}{R_{14}} + \frac{1}{R_{35}}} = \frac{1}{\frac{1}{10} + \frac{1}{14}} = 5.833 \, \Omega \\
			      E_{455'} = R_{1435} \cdot \left(\frac{E_4}{R_{14}} - \frac{E_{55'}}{R_{35}}\right) = 5.833 \cdot \left(\frac{3.9}{10} - \frac{6.65}{14}\right) = -0.496 \, \text{В} \\
		      \end{gathered}
	      \]
	\item Схема сведена к одноконтурной относительно ветви с искомым током. Искомый ток $I2$ определим с использованием ЗКII:
	      \[
		      \begin{gathered}
			      I_2 \cdot (R_2 + R_{1345}) = E_2 + E_{455'} \Leftrightarrow I_2 = \frac{E_2 + E_{455'}}{R_2 + R_{1345}} \\
			      I_2 = \frac{34.5 - 0.496}{6+5.833} = 2.874 \, \text{А}
		      \end{gathered}
	      \]

\end{enumerate}


\subsubsection{График апериодического переходного процесса}
% time	V(N003,N001)  V(n002)  V(n003)  I(C1)

\begin{figure}[H]
	\centering
	\begin{tikzpicture}[scale=0.88]
		\begin{axis}[
			width=17cm, height=12cm,
			xlabel={Время [мс]},
			ylabel={Напряжение [В]},
			grid=major,           % Enable grid
			legend style={at={(0.857,0.95)}, anchor=north west}, % Legend position
			thick,                % Line thickness
			xmin=0, xmax=600,    % X-axis range
			ymin=-4, ymax=4,      % Y-axis range
			axis y line=left,
			axis x line=bottom,
			label style={font=\small},
			tick label style={font=\small},
			major tick length=0.2cm,
			ytick style={thick, black},
			xtick style={thick, black},
			ytick={-3, -2, -1, 0, 1, 2, 3},
			extra y ticks={-4, 4},
			extra y tick style={tick style={opacity=0}},
			x unit=0.001,
			]

			% Plot V(n001)
			\addplot[
				color=teal,
				line width=1.2pt,
			] table [x expr=\thisrowno{0}*1000, y index=3, col sep=space] {./data/lab2-3.txt};
			\addlegendentry{$E$}

			% Plot V(n002)
			\addplot[
				color=orange,
				line width=1.2pt,
			] table [x expr=\thisrowno{0}*1000, y index=1, col sep=space] {./data/lab2-3.txt};
			\addlegendentry{$U_c$}


			% Plot V(n002)
			\addplot[
				color=magenta,
				line width=1.2pt,
				% nodes near coords,
				% mark=*,
				% every node near coord/.append style={font=\tiny, black}
			] table [x expr=\thisrowno{0}*1000, y index=2, col sep=space] {./data/lab2-3.txt};
			\addlegendentry{$U_L$}

			\addplot[color=black, only marks, mark=*, mark size=2pt] coordinates {(60, 0.027)};
			\node[anchor=east] at (axis cs:61,-0.15) {\scriptsize $0.027 \, \text{В} = U_L(\infty)$};

		\end{axis}

		% Secondary axis for Current
		\begin{axis}[
			width=17cm, height=12cm,
			xmin=0, xmax=600,    % X-axis range (same as for Voltage)
			ymin=-80, ymax=80,  % Y-axis range for Current
			axis y line=right,         % Place this axis on the right for Current
			axis x line=none,
			thick,
			ylabel={Ток [мА]},
			label style={font=\small},
			tick label style={font=\small},
			major tick length=0.2cm,
			ytick style={thick, black},
			ytick={-60, -40, -20, 0, 20, 40, 60},
			extra y ticks={-80, 80},
			extra y tick style={tick style={opacity=0}},
			legend style={at={(0.961,0.765)}, anchor=north east},
			]
			% Plot I
			\addplot[
				color=cyan,
				line width=1.2pt,
			] table [x expr=\thisrowno{0}*1000, y expr=\thisrowno{4}*1000, col sep=space] {./data/lab2-3.txt};
			\addlegendentry{$I$}

			\addplot[color=black, only marks, mark=*, mark size=1.5pt] coordinates {(300, 0)};
			\node[
				anchor=east,
				font=\scriptsize
			] at (axis cs:121,-47) {$-49.324\, \text{mA} = I(0+)$};
			%
			% \addplot[color=black, only marks, mark=*, mark size=1.5pt] coordinates {(60.1, 49.325)};
			% \draw[dashed, grey, very thin] (axis cs:60, 49.3) -- (axis cs:120, 49.3);
			% \node[anchor=west] at (axis cs:60.1,47) {\scriptsize $I(\infty) = 49.325 \, mA$};
		\end{axis}
	\end{tikzpicture}
\end{figure}


\subsubsection{Таблица результатов 4.4}


% -------------------------------


\subsection{Исследование колебательного процесса}

\subsubsection{Схема исследуемой цепи}
На рисунке 4 представлена схема замещения генератора прямоугольного напряжения с резистивной, ёмкостной и индуктивной нагрузками, созданная в приложении LTspice.

\begin{figure}[H]
	\centering
	\includegraphics[width=0.96\textwidth]{./data/rcl_2-schema.png}
	\caption{Схема замещения RCL-цепи в LTspice.}
\end{figure}

\subsubsection{Расчётные формулы и расчёты}
\begin{enumerate}
	\item Коэффициент затухания:

	      \[
		      \delta = \frac{R}{2L} = \frac{20}{2 \cdot 0,48} = 20,833 \, \text{с}^{-1}
	      \]

	\item Резонансная частота:

	      \[
		      \omega_c = \sqrt{\frac{1}{LC} - \delta^2} = \sqrt{\frac{1}{0,48 \cdot 300 \cdot 10^{-6}} - \frac{125^2}{6^2}} \approx 80,687 \, \text{с}^{-1}
	      \]

	\item Эксперементальное определение коэффициента затухания:

	      \[
		      \delta^* = \frac{\ln{\left(\frac{I_{m1}}{I_{m2}}\right)}}{T} = \frac{\ln{\left(\frac{0,071082}{0,031584}\right)}}{0,0781} = 10,386 \, \text{с}^{-1}
	      \]

	\item Эксперементальное определение резонансной частоты:

	    \[
	        \omega_c^* = \frac{2\pi}{T} = \frac{2\pi}{0,0781} = 80,451 \, \text{с}^{-1}
	    \]
\end{enumerate}


\subsubsection{График колебательного переходного процесса}
% time	V(N003,N001)  V(n002)  V(n003)  I(C1)

\begin{figure}[H]
	\centering
	\begin{tikzpicture}[scale=0.88]
		\begin{axis}[
				width=17cm, height=12cm,
				xlabel={Time [ms]},    % X-axis label
				ylabel={Voltage [V]},    % Y-axis label
				grid=major,           % Enable grid
				legend style={at={(0.857,0.95)}, anchor=north west}, % Legend position
				thick,                % Line thickness
				xmin=0, xmax=600,    % X-axis range
				ymin=-4, ymax=4,      % Y-axis range
				axis y line=left,
				axis x line=bottom,
				label style={font=\small},
				tick label style={font=\small},
				major tick length=0.2cm,
				ytick style={thick, black},
				xtick style={thick, black},
				ytick={-3.2, -2.4, -1.6, -0.8, 0, 0.8, 1.6, 2.4, 3.2},
				extra y ticks={-4, 4},
				extra y tick style={tick style={opacity=0}},
				x unit=0.001,
			]

			% Plot V(n001)
			\addplot[
				color=red,
				thick,
			] table [x expr=\thisrowno{0}*1000, y index=3, col sep=space] {./data/lab2-4.txt};
			\addlegendentry{$E$}

			% Plot V(n002)
			\addplot[
				color=magenta,
				thick,
			] table [x expr=\thisrowno{0}*1000, y index=1, col sep=space] {./data/lab2-4.txt};
			\addlegendentry{$U_L$}
			%
			% \addplot[color=black, only marks, mark=*, mark size=1pt] coordinates {(0, 4)};
			% \node[anchor=west] at (axis cs:0,3.85) {\scriptsize $U_L(0+) = 4 \, V$};
			%
			% \addplot[color=black, only marks, mark=*, mark size=1.5pt] coordinates {(60, 0.027)};
			% \node[anchor=east] at (axis cs:61,-0.15) {\scriptsize $0.027 \, V = U_L(\infty)$};
			% \draw[dashed, grey, very thin] (axis cs:0, 0.027) -- (axis cs:60, 0.027);

			% Plot V(n002)
			\addplot[
				color=green,
				thick,
				% nodes near coords,
				% mark=*,
				% every node near coord/.append style={font=\tiny, black}
			] table [x expr=\thisrowno{0}*1000, y index=2, col sep=space] {./data/lab2-4.txt};
			\addlegendentry{$U_c$}

			% \draw[dashed,grey,very thin] (axis cs:120,1.972884) -- (axis cs:60.1,1.972884);
			% \addplot[color=black, only marks, mark=*, mark size=1.5pt] coordinates {(60.1,1.972884)};
			% \node[anchor=west] at (axis cs:60.1,1.872884) {\scriptsize $U_c(\infty) = 1.973 \, V$};
			%
			% \draw[dashed,grey,very thin] (axis cs:120,-1.973) -- (axis cs:120, -1.973)
			% \addplot[color=black, only marks, mark=*, mark size=1.5pt] coordinates {(120, -1.973)};
			% \node[anchor=east] at (axis cs:121,-1.82) {\scriptsize $ -1.973 \, V = U_c(0+)$};


		\end{axis}

		% Secondary axis for Current
		\begin{axis}[
				width=17cm, height=12cm,
				xmin=0, xmax=600,    % X-axis range (same as for Voltage)
				ymin=-80, ymax=80,  % Y-axis range for Current
				axis y line=right,         % Place this axis on the right for Current
				axis x line=none,
				thick,
				ylabel={Current [mA]}, % Y-axis label for Current
				label style={font=\small},
				tick label style={font=\small},
				major tick length=0.2cm,
				ytick style={thick, black},
				ytick={-70, -60, -50, -40, -30, -20, -10, 0, 10, 20, 30, 40, 50, 60, 70},
				extra y ticks={-80, 80},
				extra y tick style={tick style={opacity=0}},
				legend style={at={(0.961,0.765)}, anchor=north east},
			]
			% Plot I
			\addplot[
				color=cyan,
				thick,
			] table [x expr=\thisrowno{0}*1000, y expr=\thisrowno{4}*1000, col sep=space] {./data/lab2-4.txt};
			\addlegendentry{$I$}

			% \addplot[color=black, only marks, mark=*, mark size=1.5pt] coordinates {(120, -49.324)};
			% \node[
			% 	anchor=east,
			% 	font=\scriptsize
			% ] at (axis cs:121,-47) {$-49.324\, \text{mA} = I(0+)$};
			%
			% \addplot[color=black, only marks, mark=*, mark size=1.5pt] coordinates {(60.1, 49.325)};
			% \draw[dashed, grey, very thin] (axis cs:60, 49.3) -- (axis cs:120, 49.3);
			% \node[anchor=west] at (axis cs:60.1,47) {\scriptsize $I(\infty) = 49.325 \, mA$};
		\end{axis}
	\end{tikzpicture}
\end{figure}


\subsubsection{Таблица результатов 4.5}

\subsection{Выводы по второй части}
