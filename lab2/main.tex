\documentclass[a4paper]{article}

% Packages
\usepackage[14pt]{extsizes}
\usepackage[T2A]{fontenc}
\usepackage[russian]{babel}
\usepackage[left=20mm, top=15mm, right=15mm, bottom=20mm]{geometry}
\usepackage{graphicx} % For images
\usepackage{amsmath, amssymb} % For equations
\usepackage{booktabs} % For better tables
\usepackage{pgfplots} % For plotting graphs
\usepackage{xcolor} % For color support
\usepackage{caption} % For captioning tables and figures
\usepackage{float} % For precise float placement (images, tables)
\usepackage{array} % For better table management
\usepackage[hidelinks]{hyperref} % For table of contents to be clickable
\usepackage{bookmark}
\usepackage{multirow}
\usepackage{array}
\pgfplotsset{compat=1.17}

% Importing custom definitions (lstset, tikzset, etc.)
\input{../common/title-lab.tex}
\input{../common/title-hw.tex}


% -------------------------------

\begin{document}

% Title page
\labtitle{2}{Исследование переходных процессов в электрических цепях}{3331}{9}{Дворкин Борис Александрович}{25.09.2024}{09.10.2024}{09.10.2024}{}
\thispagestyle{empty}

\newpage
\pagestyle{plain}
\setcounter{page}{1}

% -------------------------------

% autogenerated table of contents
\linespread{0.9}
\tableofcontents
\linespread{1}

% -------------------------------

\newpage
\section*{Цель работы}
\addcontentsline{toc}{section}{Цель работы}

Исследование переходных процессов в электрических цепях первого и
второго порядков с источником постоянного и переменного напряжения.

% -------------------------------

\section*{Часть 1}
\addcontentsline{toc}{section}{Часть 1}
\stepcounter{section}
\subsection{Введение}
В данной части лабораторной работы произведены измерения действующих значений входного напряжения, тока и фазового сдвига между ними для девяти различных двухполюсников, а также произведены сравнения результатов с расчётными значениями.

\subsection{Параметры источника}

\subsection{Общие расчёты}
\begin{enumerate}
	\item Угловая частота:
	      \[
		      \omega = 2 \pi f = 2 \cdot 3.1416 \cdot 19.894 \approx 125 \, \text{рад/с} \\
	      \]

	\item Реактивная составляющая сопротивления ёмкостного элемента:
	      \[
		      X_c = \frac{1}{\omega C} = \frac{1}{125 \cdot 71.454 \cdot 10^{-6}} = 111.96 \, \text{Ом}
	      \]

	\item Реактивная составляющая сопротивления индуктивного элемента:
	      \[
		      X_L = \omega L = 125 \cdot 23.094 \cdot 10^{-3} = 2.887 \, \text{Ом}
	      \]

	\item Реактивная проводимость ёмкостного элемента:
	      \[
		      B_c = \omega C = 125 \cdot 71.454 \cdot 10^{-6} = 0.00893 \, \text{См}
	      \]

	\item Реактивная проводимость индуктивного элемента:
	      \[
		      B_k = \frac{X_L}{R_k^2 + X_L^2} = \frac{2.887}{5^2 + (2.887)^2} = 0.0866 \, \text{См}
	      \]
\end{enumerate}


\subsection{Двухполюсник 1}
\subsubsection{Схема исследуемой цепи}
dgfdgd
\subsubsection{Расчётные формулы и расчёты}
\subsubsection{Векторная диаграмма входного напряжения и тока}

\subsection{Двухполюсник 2}
\subsubsection{Схема исследуемой цепи}
\subsubsection{Расчётные формулы и расчёты}
\subsubsection{Векторная диаграмма входного напряжения и тока}

\subsection{Двухполюсник 3}
\subsubsection{Схема исследуемой цепи}
\subsubsection{Расчётные формулы и расчёты}
\subsubsection{Векторная диаграмма входного напряжения и тока}

\subsection{Двухполюсник 4}
\subsubsection{Схема исследуемой цепи}
\subsubsection{Расчётные формулы и расчёты}
\subsubsection{Векторная диаграмма входного напряжения и тока}

\subsection{Двухполюсник 5}
\subsubsection{Схема исследуемой цепи}
\subsubsection{Расчётные формулы и расчёты}
\subsubsection{Векторная диаграмма входного напряжения и тока}

\subsection{Двухполюсник 6}
\subsubsection{Схема исследуемой цепи}
\subsubsection{Расчётные формулы и расчёты}
\subsubsection{Векторная диаграмма входного напряжения и тока}

\subsection{Двухполюсник 7}
\subsubsection{Схема исследуемой цепи}
\subsubsection{Расчётные формулы и расчёты}
\subsubsection{Векторная диаграмма входного напряжения и тока}

\subsection{Двухполюсник 8}
\subsubsection{Схема исследуемой цепи}
\subsubsection{Расчётные формулы и расчёты}
\subsubsection{Векторная диаграмма входного напряжения и тока}

\subsection{Двухполюсник 9}
\subsubsection{Схема исследуемой цепи}
\subsubsection{Расчётные формулы и расчёты}
\subsubsection{Векторная диаграмма входного напряжения и тока}

\subsection{Заполненная таблица 2.2}

\subsection{Выводы}


% -------------------------------

\section*{Часть 2}
\addcontentsline{toc}{section}{Часть 2}
\stepcounter{section}
\subsection{Апериодический процесс}

\subsubsection{Схема исследуемой цепи}
На рисунке 1.1 представлена схема замещения источника электрической энергии постоянного тока и нагрузки, созданная в приложении LTspice.

% \begin{figure}[H]
% 	\centering
% 	\includegraphics[width=0.6\textwidth]{rcl-schema.png} % Make sure the path to the image is correct
% 	\caption{Схема замещения источника электрической энергии в LTspice.}
% \end{figure}

\subsubsection{Расчётные формулы и расчёты}

\subsubsection{Графики переходных процессов}

\subsubsection{Таблица результатов 4.4}

\subsection{Колебательный процесс}

\subsubsection{Схема исследуемой цепи}
На рисунке 1.1 представлена схема замещения источника электрической энергии постоянного тока и нагрузки, созданная в приложении LTspice.

% \begin{figure}[H]
% 	\centering
% 	\includegraphics[width=0.6\textwidth]{rcl-schema.png} % Make sure the path to the image is correct
% 	\caption{Схема замещения источника электрической энергии в LTspice.}
% \end{figure}

\subsubsection{Расчётные формулы и расчёты}

\subsubsection{Графики переходных процессов}

\subsubsection{Таблица результатов 4.5}

\subsection{Выводы по второй части}


% % -------------------------------
%
% \section*{Выводы по работе}
% \addcontentsline{toc}{section}{Выводы по работе}
% \stepcounter{section}
%
% Вывод по работе

\end{document}
