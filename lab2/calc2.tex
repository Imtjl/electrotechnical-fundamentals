\begin{enumerate}
	\item Постоянная времени:

	      \[
		      \tau = \frac{L}{R} = \frac{0.48}{40} = 0.012 \, \text{с} = 12 \, \text{мс}
	      \]

	\item Ток после коммутации:

	      \[
		      \lim_{t \to 0^+} I(t) = \lim_{t \to 0^-} I(t) = \lim_{t \to 0^-} \frac{E(t)}{R} = \frac{-2}{40} = -0.05 \, \text{А} = -50 \, \text{мА}
	      \]

	\item Напряжение на катушке после коммутации:

	      \[
		      \lim_{t \to 0^+} U_L(t) = \lim_{t \to 0^+} E(t) - \lim_{t \to 0^-} I(t) \cdot R = 2 - (-0.05 \cdot 40) = 2 + 2 = 4 \, \text{В}
	      \]

	\item Установившийся ток:

	      \[
		      \lim_{t \to \infty} I(t) = \frac{E}{R} = \frac{2}{40} = 0.05 \, \text{А} = 50 \, \text{мА}
	      \]

	\item Напряжение на катушке в установившемся режиме:

	      \[
		      \lim_{t \to \infty} U_L(t) = \lim_{t \to \infty} I(t) \cdot R_k = 50 \cdot 0 = 0 \, \text{В}
	      \]
\end{enumerate}

Значение времени переходного процесса \( t_{0.5} \) определяется как время, за которое напряжение достигает половины своего амплитудного значения. Постоянная времени \( \tau \) определяется как:

\[
	\tau = \frac{t_{0.5}}{\ln 2}
\]

\textbf{Вычисление постоянной времени \( \tau \):}

\[
	\tau = \frac{t_{0.5_{U_L}}}{\ln 2} = \frac{8.418331 \, \text{мс}}{0.69314718} = 12.143 \, \text{мс}
\]

Постоянная времени \( \tau \) была определена экспериментально и составляет
\( 12.144 \, \text{мс} \). Это значение будет использовано для расчёта соответствующих токов и напряжений в цепи.
