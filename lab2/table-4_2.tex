Таблица 4.2 содержит эксперементальные и расчётные результаты длительности переходного процесса в RC-цепи, а также тока и напряжения в момент коммутации и в установившемся режиме.

\begin{table}[h]
	\centering
	\resizebox{\textwidth}{!}{  % Scale to fit within the text width
		\begin{tabular}{|c|c|c|c|c|c|c|c|}
			\hline
			$R$ [Ом]            & $C$ [мкФ]            & Тип данных & $I(0+)$ [мА] & $I(\infty)$ [мА] & $U_C(0+)$ [В] & $U_C(\infty)$ [В] & $\tau$ [мкс] \\
			\hline
			\multirow{2}{*}{40} & \multirow{2}{*}{300} & эксп.      & 99.996       & 0.674            & -1.999        & 1.973             & 12\,144      \\
			\cline{3-8}
			                    &                      & расч.      & 100          & 0                & -2            & 2                 & 12\,000      \\
			\hline
		\end{tabular}
	}
	\caption{Результаты измерений и расчётов для RC-цепи}
\end{table}
