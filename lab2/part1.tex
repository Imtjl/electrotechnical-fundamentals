\subsection{Исследование RC-цепи}

\subsubsection{Схема исследуемой цепи}
На рисунке 1.1 представлена схема замещения генератора прямоугольного напряжения с резистивной и ёмкостной нагрузкой, созданная в приложении LTspice.

\begin{figure}[H]
	\centering
	\includegraphics[width=1\textwidth]{./data/rc-schema.png}
	\caption{Схема замещения RC-цепи в LTspice.}
\end{figure}

\subsubsection{Расчётные формулы и расчёты}
\begin{enumerate}
	\item Определение топологии цепи:
	      \begin{align*}
		      p^*           & = 6 \, (\text{общее количество ветвей}),                                     \\
		      p_{\text{ит}} & = 1 \, (\text{количество ветвей с источниками тока}),                        \\
		      p             & = p^* - p_{\text{ит}} = 6 - 1 = 5 \, (\text{количество неизвестных токов}),  \\
		      q             & = 4 \, (\text{количество узлов}),                                            \\
		      n             & = p - (q - 1) = 5 - (4 - 1) = 2 \, (\text{количество независимых контуров}), \\
		      m_I           & = q - 1 = 4 - 1 = 3 \, (\text{количество уравнений по ЗКИ}),                 \\
		      m_{II}        & = n = 2 \, (\text{количество уравнений по ЗКП}).
	      \end{align*}

	\item Произвольно обозначим $p$ неизвестных токов, $q$ узлов и $n$ независимых контуров. В любом месте ветви обозначим стрелки и имена искомых токов. В узлах поставим порядковый номер (арабская цифра), обведённый окружностью. Для выбранных контуров укажем направление их обхода, и внутри контура укажем порядковый номер (римская цифра), обведённый окружностью.
	      \begin{figure}[H]
	\centering
	\begin{circuitikz}[american, scale=1.15]

		\draw
		(0,0)
		to[R, l=$R_3$, i=$I_3$] (10,0)
		-- (10, -3)
		to[short, *-] (10, -4.1);

		\draw
		(10, -3)
		to[R, l=$R_5$, i=$I_5$] (7.5, -3)
		to[I, l=$E_5$] (5, -3)
		to[short, *-] (5, -3)
		to[R, l=$R_2$, i=$I_2$] (2.5, -3)
		to[I, l=$E_2$] (0, -3);

		\draw
		(0,0)
		-- (0,-3)
		to[short, *-] (0, -3)
		to[R, l=$R_1$, i=$I_1$] (0, -6)
		-- (5, -6) to[short, *-] (10, -6)
		-- (10, -4.9);

		\draw
		(5, -3)
		to[R, l=$R_4$, i=$I_4$] (5, -6);

		\draw
		(10,-4.5) circle(0.4cm)
		(10,-4.45) edge[thick, -{Straight Barb[length=2mm]}] (10, -4.44)
		(10,-4.35) edge[thick, -{Straight Barb[length=2mm]}] (10, -4.34)
		node at (10.75, -4.5) {$J_6$};

		\draw[
			color=red!60, thick, dashed,
			postaction={decorate,decoration={
							markings,
							mark=at position 0.25 with {\arrow[scale=1.5,fill=red!90]{<}},
							mark=at position 0.5 with {\arrow[scale=1.5,fill=red!90]{<}},
							mark=at position 0.75 with {\arrow[scale=1.5,fill=red!90]{<}},
							mark=at position 1 with {\arrow[scale=1.5,fill=red!90]{<}},
						}}
		]
		(5, -1.5) ellipse [x radius=4cm, y radius=0.97cm]
		node[scale=1.2, draw, thick, solid, circle] {$II$};

		\draw[
			color=red!60, thick, dashed,
			postaction={decorate,decoration={
							markings,
							mark=at position 0.25 with {\arrow[scale=1.5,fill=red!90]{>}},
							mark=at position 0.5 with {\arrow[scale=1.5,fill=red!90]{>}},
							mark=at position 0.75 with {\arrow[scale=1.5,fill=red!90]{>}},
							mark=at position 1 with {\arrow[scale=1.5,fill=red!90]{>}},
						}}
		]
		(2.7, -4.7) ellipse [x radius=1.5cm, y radius=1cm]
		node[scale=1.2, draw, thick, solid, circle] {$I$};

		\draw (0, -3) node[draw, circle, fill=blue!10, scale=0.8] {\textcolor{blue}{1}};
		\draw (5, -3) node[draw, circle, fill=blue!10, scale=0.8] {\textcolor{blue}{2}};
		\draw (10, -3) node[draw, circle, fill=blue!10, scale=0.8] {\textcolor{blue}{3}};
		\draw (5, -6) node[draw, circle, fill=blue!10, scale=0.8] {\textcolor{blue}{4}};

	\end{circuitikz}
	\caption{Схема электрической цепи с контурами, узлами и токами}
\end{figure}


	\item Составим систему из $m_I$ уравнений по ЗКI и $m_{II}$ уравнений по ЗКII:
	      \[
		      \begin{cases}
			      \text{ЗК1:} & \begin{cases}
				                    I_2 - I_1 - I_3 = 0, \quad \text{для узла 1}, \\
				                    I_5 - I_4 - I_2 = 0, \quad \text{для узла 2}, \\
				                    I_3 - I_5 = -J_6, \quad \text{для узла 3},
			                    \end{cases}                        \\[10pt]
			      \text{ЗК2:} & \begin{cases}
				                    I_2 R_2 + I_1 R_1 - I_4 R_4 = E_2, \quad \text{для контура I}, \\
				                    I_3 R_3 + I_5 R_5 + I_2 R_2 = E_5 + E_2, \quad \text{для контура II}.
			                    \end{cases}
		      \end{cases}
	      \]

	\item Представим в матричной форме \( A \cdot X = F \):
	      \[
		      \begin{bmatrix}
			      -1  & 1   & -1  & 0    & 0   \\
			      0   & -1  & 0   & -1   & 1   \\
			      0   & 0   & 1   & 0    & -1  \\
			      R_1 & R_2 & 0   & -R_4 & 0   \\
			      0   & R_2 & R_3 & 0    & R_5
		      \end{bmatrix}
		      \cdot
		      \begin{bmatrix}
			      I_1 \\
			      I_2 \\
			      I_3 \\
			      I_4 \\
			      I_5
		      \end{bmatrix}
		      =
		      \begin{bmatrix}
			      0    \\
			      0    \\
			      -J_6 \\
			      E_2  \\
			      E_5 + E_2
		      \end{bmatrix}
	      \]

	\item Подставим численные значения:
	      \[
		      \begin{bmatrix}
			      -1 & 1  & -1 & 0  & 0  \\
			      0  & -1 & 0  & -1 & 1  \\
			      0  & 0  & 1  & 0  & -1 \\
			      8  & 6  & 0  & -2 & 0  \\
			      0  & 6  & 7  & 0  & 7
		      \end{bmatrix}
		      \cdot
		      \begin{bmatrix}
			      I_1 \\
			      I_2 \\
			      I_3 \\
			      I_4 \\
			      I_5
		      \end{bmatrix}
		      =
		      \begin{bmatrix}
			      0     \\
			      0     \\
			      -1,95 \\
			      34,5  \\
			      41,5
		      \end{bmatrix}
	      \]
	\item Решим систему уравнений:
	      \[
		      X
		      =
		      \begin{pmatrix}
			      I_1 \\
			      I_2 \\
			      I_3 \\
			      I_4 \\
			      I_5
		      \end{pmatrix}
		      =
		      A^{-1} \cdot F =
		      \begin{pmatrix}
			      2.116  \\
			      2.874  \\
			      0.758  \\
			      -0.166 \\
			      2.708
		      \end{pmatrix}
	      \]
\end{enumerate}


\subsubsection{График переходного процесса в RC-цепи}
На графике показан переходной процесс в RC-цепи. Красная линия — это напряжение источника \( E \), зелёная — напряжение на конденсаторе \( U_C \), а синяя — ток \( I \) через цепь.

\begin{enumerate}[noitemsep,topsep=0pt,left=6pt,label=\arabic*.]
	\item В момент коммутации напряжение на конденсаторе остаётся на уровне \( U_C(0^+) = -1.999 \, \text{В} \), так как напряжение на конденсаторе не может измениться мгновенно.
	\item Ток через цепь \( I(0^+) \) сразу после коммутации резко возрастает до \( I(0^+) = 99.996 \, \text{мА} \), что видно на графике, так как ток в RC-цепи может изменяться мгновенно.
	\item Постепенно ток начинает снижаться, а напряжение на конденсаторе увеличивается, стремясь к установившемуся значению.
	\item В установившемся режиме \( t \to \infty \), напряжение на конденсаторе достигает \( U_C(\infty) = 1.973 \, \text{В} \), а ток в цепи \( I(\infty) = 0.674 \, \text{мА} \), что подтверждается графиком.
\end{enumerate}

Таким образом, график иллюстрирует процесс заряда конденсатора в RC-цепи, где конденсатор препятствует мгновенному изменению напряжения, а сопротивление влияет на скорость затухания переходного процесса.
% lab2.txt data structure:
% time	V(n001)	V(n002)	I(R1)

\begin{figure}[h]
	\centering
	\begin{tikzpicture}[scale=0.88]

		% Voltage axis
		\begin{axis}[
				width=17cm, height=12cm,
				xlabel={Time [ms]},    % X-axis label
				ylabel={Voltage [V]},    % Y-axis label
				grid=major,           % Enable grid
				legend style={at={(0.857,0.95)}, anchor=north west}, % Legend position
				thick,                % Line thickness
				xmin=0, xmax=120,    % X-axis range
				ymin=-2, ymax=2,      % Y-axis range
				axis y line=right,
				axis x line=bottom,
				label style={font=\small},
				tick label style={font=\small},
				major tick length=0.2cm,
				ytick style={thick, black},
				xtick style={thick, black},
				ytick={-1.6, -1.2, -0.8, -0.4, 0, 0.4, 0.8, 1.2, 1.6},
				extra y ticks={-2, 2},
				extra y tick style={tick style={opacity=0}},
				x unit=0.001,
			]

			% Plot V(n001)
			\addplot[
				color=red,
				thick,
			] table [x expr=\thisrowno{0}*1000, y index=1, col sep=space] {./data/lab2.txt};
			\addlegendentry{$E$}

			% Plot V(n002)
			\addplot[
				color=green,
				thick,
				% nodes near coords,
				% mark=*,
				% every node near coord/.append style={font=\tiny, black}
			] table [x expr=\thisrowno{0}*1000, y index=2, col sep=space] {./data/lab2.txt};
			\addlegendentry{$U_c$}

			\draw[dashed,grey,very thin] (axis cs:120,1.972884) -- (axis cs:60.1,1.972884);
			\addplot[color=black, only marks, mark=*, mark size=1.5pt] coordinates {(60.1,1.972884)};
			\node[anchor=west] at (axis cs:60.1,1.872884) {\scriptsize $U_c(\infty) = 1.973 \, V$};

			\draw[dashed,grey,very thin] (axis cs:120,-1.973) -- (axis cs:120, -1.973)
			\addplot[color=black, only marks, mark=*, mark size=1.5pt] coordinates {(120, -1.973)};
			\node[anchor=east] at (axis cs:121,-1.82) {\scriptsize $ -1.973 \, V = U_c(0+)$};

		\end{axis}

		% Current axis
		\begin{axis}[
				width=17cm, height=12cm,
				xmin=0, xmax=120,    % X-axis range (same as for Voltage)
				ymin=-100, ymax=100,  % Y-axis range for Current
				axis y line=left,         % Place this axis on the right for Current
				axis x line=none,
				thick,
				ylabel={Current [mA]}, % Y-axis label for Current
				label style={font=\small},
				tick label style={font=\small},
				major tick length=0.2cm,
				ytick style={thick, black},
				ytick={-80, -60, -40, -20, 0, 20, 40, 60, 80},  % Y-axis ticks without ymax
				extra y ticks={-100, 100}, % Add min and max ticks for right axis
				extra y tick style={tick style={opacity=0}}, % Hide tick marks for these
				legend style={at={(0.955,0.82)}, anchor=north east}, % Legend position for Current
			]
			% Plot I(R1)
			\addplot[
				color=blue,
				thick,
				% mark=ball,
				% mark size=1.5pt
			] table [x expr=\thisrowno{0}*1000, y expr=\thisrowno{3}*1000, col sep=space] {./data/lab2.txt};
			\addlegendentry{$I$}

			\addplot[color=black, only marks, mark=*, mark size=1pt] coordinates {(0.13, 100)};
			\node[anchor=west] at (axis cs:0,95.5) {\scriptsize $I(0+) = 99.996 \, mA$};

			\addplot[color=black, only marks, mark=*, mark size=1.5pt] coordinates {(60, 0.674)};
			\node[anchor=east] at (axis cs:61,-4.5) {\scriptsize $0.674 \, mA = I(\infty)$};

			\addplot[color=black, only marks, mark=*, mark size=1.7pt] coordinates {(8.4, 50)};
			\node[anchor=west] at (axis cs:8.4,50) {\scriptsize $I(t_{0.5}) = 50\, \text{mA}$};

			\addplot[color=black, only marks, mark=*, mark size=1.5pt] coordinates {(8.4, -100)};
			\node[anchor=east] at (axis cs:9,-95) {\small $t_{0.5}$};

			\draw[dashed, grey, very thin] (axis cs:8.4, 50) -- (axis cs:8.4, -102);

		\end{axis}
	\end{tikzpicture}
\end{figure}


\subsubsection{Таблица результатов 4.2}
Таблица 4.2 содержит эксперементальные и расчётные результаты длительности переходного процесса в RC-цепи, а также тока и напряжения в момент коммутации и в установившемся режиме.

\begin{table}[h]
	\centering
	\resizebox{\textwidth}{!}{  % Scale to fit within the text width
		\begin{tabular}{|c|c|c|c|c|c|c|c|}
			\hline
			$R$ [Ом]            & $C$ [мкФ]            & Тип данных & $I(0+)$ [мА] & $I(\infty)$ [мА] & $U_C(0+)$ [В] & $U_C(\infty)$ [В] & $\tau$ [мкс] \\
			\hline
			\multirow{2}{*}{40} & \multirow{2}{*}{300} & эксп.      & 99.996       & 0.674            & -1.999        & 1.973             & 12\,144      \\
			\cline{3-8}
			                    &                      & расч.      & 100          & 0                & -2            & 2                 & 12\,000      \\
			\hline
		\end{tabular}
	}
	\caption{Результаты измерений и расчётов для RC-цепи}
\end{table}



% -------------------------------

\newpage
\subsection{Исследование RL-цепи}

\subsubsection{Схема исследуемой цепи}
На рисунке 1.1 представлена схема замещения генератора прямоугольного напряжения с активно-индуктивной нагрузкой, созданная в приложении LTspice.

\begin{figure}[H]
	\centering
	\includegraphics[width=1\textwidth]{./data/rl-schema.png}
	\caption{Схема замещения RL-цепи в LTspice.}
\end{figure}

\subsubsection{Расчётные формулы и расчёты}
\begin{enumerate}
	\item Расчёт действующего тока в цепи:
	      \[
		      \begin{gathered}
			      I = \frac{U}{Z} = \frac{U}{\sqrt{R^2 + X^2}} \\
			      X = -X_C, R = 0 \implies I = \frac{U}{X_C} = \frac{6}{111.96} = 0.0536 \, \text{А}
		      \end{gathered}
	      \]
	\item Расчёт фазового сдвига:
	      \[
		      \phi = \arctan\left(-\inf\right) = -90^{\circ}
	      \]
\end{enumerate}


\subsubsection{График переходного процесса в RL-цепи}
На графике показан переходной процесс в RL-цепи. Красная линия — это напряжение источника \( E \), зелёная — напряжение на катушке \( U_L \), а синяя — ток \( I \) через цепь.

\begin{enumerate}[noitemsep,topsep=0pt,left=6pt,label=\arabic*.]
	\item В момент коммутации \( t = 0 \), напряжение на катушке резко возрастает до \( U_L(0^+) = 4 \, \text{В} \), что подтверждается на графике.
	\item Ток в цепи сразу после коммутации составляет \( I(0^+) = -49.998 \, \text{мА} \), после чего ток начинает постепенно увеличиваться.
	\item По мере того как ток через цепь нарастает, напряжение на катушке уменьшается и стремится к нулю. Это происходит за время, характеризуемое постоянной времени RL-цепи.
	\item В установившемся режиме \( t \to \infty \), напряжение на катушке стремится к \( U_L(\infty) = 0.027 \, \text{В} \), а ток достигает значения \( I(\infty) = 49.325 \, \text{мА} \), что можно увидеть на графике.
\end{enumerate}

Таким образом, график иллюстрирует затухание переходного процесса в RL-цепи, где индуктивность препятствует мгновенному изменению тока, а сопротивление и индуктивность совместно влияют на временные характеристики процесса.
% lab2-2.txt graph structure:
% time	V(n001)	V(n002)	I(R1)

\begin{figure}[H]
	\centering
	\begin{tikzpicture}[scale=0.88]
		\begin{axis}[
				width=17cm, height=12cm,
				xlabel={Time [ms]},    % X-axis label
				ylabel={Voltage [V]},    % Y-axis label
				grid=major,           % Enable grid
				legend style={at={(0.857,0.95)}, anchor=north west}, % Legend position
				thick,                % Line thickness
				xmin=0, xmax=120,    % X-axis range
				ymin=-4, ymax=4,      % Y-axis range
				axis y line=left,
				axis x line=bottom,
				label style={font=\small},
				tick label style={font=\small},
				major tick length=0.2cm,
				ytick style={thick, black},
				xtick style={thick, black},
				ytick={-3.2, -2.4, -1.6, -0.8, 0, 0.8, 1.6, 2.4, 3.2},
				extra y ticks={-4, 4},
				extra y tick style={tick style={opacity=0}},
				x unit=0.001,
			]

			% Plot V(n001)
			\addplot[
				color=red,
				line width=1.2pt,
			] table [x expr=\thisrowno{0}*1000, y index=1, col sep=space] {./data/lab2-2.txt};
			\addlegendentry{$E$}

			% Plot V(n002)
			\addplot[
				color=green,
				line width=1.2pt,
			] table [x expr=\thisrowno{0}*1000, y index=2, col sep=space] {./data/lab2-2.txt};
			\addlegendentry{$U_L$}

			\addplot[color=black, only marks, mark=*, mark size=1.5pt] coordinates {(0, 4)};
			\node[anchor=west] at (axis cs:0,3.85) {\scriptsize $U_L(0+) = 4 \, V$};

			\addplot[color=black, only marks, mark=*, mark size=2pt] coordinates {(60, 0.027)};
			\node[anchor=east] at (axis cs:61,-0.15) {\scriptsize $0.027 \, V = U_L(\infty)$};
			\draw[dashed, grey, very thin] (axis cs:0, 0.027) -- (axis cs:60, 0.027);

		\end{axis}

		% Secondary axis for Current
		\begin{axis}[
				width=17cm, height=12cm,
				xmin=0, xmax=120,    % X-axis range (same as for Voltage)
				ymin=-50, ymax=50,  % Y-axis range for Current
				axis y line=right,         % Place this axis on the right for Current
				axis x line=none,
				line width=1.2pt,
				ylabel={Current [mA]}, % Y-axis label for Current
				label style={font=\small},
				tick label style={font=\small},
				major tick length=0.2cm,
				ytick style={thick, black},
				ytick={-40, -30, -20, -10, 0, 10, 20, 30, 40},
				extra y ticks={-50, 50},
				extra y tick style={tick style={opacity=0}},
				legend style={at={(0.96,0.82)}, anchor=north east},
			]
			% Plot I(R1)
			\addplot[
				color=blue,
				line width=1.2pt,
			] table [x expr=\thisrowno{0}*1000, y expr=\thisrowno{3}*1000, col sep=space] {./data/lab2-2.txt};
			\addlegendentry{$I$}

			\addplot[color=black, only marks, mark=*, mark size=2pt] coordinates {(120, -49.324)};
			\node[
				anchor=east,
				font=\scriptsize
			] at (axis cs:121,-47) {$-49.324\, \text{mA} = I(0+)$};

			\addplot[color=black, only marks, mark=*, mark size=2pt] coordinates {(60.1, 49.325)};
			\draw[dashed, grey, very thin] (axis cs:60, 49.3) -- (axis cs:120, 49.3);
			\node[anchor=west] at (axis cs:60.1,47) {\scriptsize $I(\infty) = 49.325 \, mA$};
		\end{axis}
	\end{tikzpicture}
\end{figure}


\subsubsection{Таблица результатов 4.3}
Таблица 4.3 содержит эксперементальные и расчётные результаты длительности переходного процесса в RL-цепи, а также тока и напряжения в момент коммутации и в установившемся режиме.

\begin{table}[h]
	\centering
	\resizebox{\textwidth}{!}{
		\begin{tabular}{|c|c|c|c|c|c|c|c|}
			\hline
			$R$ [Ом]            & $L$ [мГн]            & Тип данных & $I(0+)$ [мА] & $I(\infty)$ [мА] & $U_L(0+)$ [В] & $U_L(\infty)$ [В] & $\tau$ [мкс] \\
			\hline
			\multirow{2}{*}{40} & \multirow{2}{*}{480} & эксп.      & -49.998      & 49.325           & 4             & 0.027             & 12\,143      \\
			\cline{3-8}
			                    &                      & расч.      & -50          & 50               & 4             & 0                 & 12\,000      \\ \hline
		\end{tabular}
	}
	\caption{Результаты измерений и расчётов для RC-цепи}
\end{table}



% -------------------------------


\subsection{Выводы по первой части}

В результате выполнения первой части лабораторной работы я исследовал переходные процессы в RC и RL цепях и выяснил, что постоянная времени \(\tau\), равная 12 мс, описывает скорость изменения токов и напряжений. За это время величины изменяются примерно на 63\% от максимального значения. Экспоненциальный характер графиков обусловлен свойствами дифференциальных уравнений, описывающих процессы зарядки и разрядки конденсатора, а также роста и спада тока через индуктивность.

Экспериментальные значения \(\tau\), полученные аппроксимацией на основе ближайших точек графика, экспортированных из LTSpice, составили 12.144 мс для RC-цепи и 12.143 мс для RL-цепи, что близко к расчётным 12 мс. Отклонение менее 1.2\% говорит о высокой точности эксперимента. Кроме того, отклонения других величин (токов и напряжений в момент коммутации и в установившемся режиме) также оказались небольшими: ток в RC-цепи составил 99.996 мА против расчётных 100 мА, а напряжение на катушке в RL-цепи было 0.027 В против теоретических 0 В в установившемся режиме. Эти результаты показывают, что эксперименты подтвердили теоретические модели с минимальными расхождениями.

При этом я заметил, что в каждом эксперименте ток и напряжение достигают половины своих амплитуд \(t_{0.5}\) за разное время (разница составляет около 4–6 мс). Это объясняется тем, что конденсатор и катушка сопротивляются резким изменениям: электростатическое поле конденсатора замедляет изменение напряжения на нём, так как при накоплении заряда на обкладках возникает сильное электростатическое поле, которое требует времени для изменения (перемещения зарядов через цепь). Катушка же сопротивляется изменению тока за счёт магнитного поля, которое индуцирует ЭДС (самоиндукция), противодействующую изменению тока (закон Ленца).

Также в будущем можно было бы учесть полное сопротивление (импеданс) катушки и конденсатора для более точного моделирования, но в наших опытах эти эффекты не рассматривались (пренебрегали реактивными сопротивлениями).
