Постоянная времени $\tau$ в RC-цепи рассчитывается по формуле:

\[
	\tau = R \cdot C
\]

где $R = 40 \, \Omega$ и $C = 300 \, \mu F$.

Подставим значения:

\[
	\tau = 40 \, \Omega \cdot 300 \cdot 10^{-6} \, F \cdot 10^{-3} = 12 \, \text{мс}
\]

Расчётные значения тока $I(0+)$ и напряжения на конденсаторе $U_C(0+)$ в момент коммутации, а также установившиеся значения $I(\infty)$ и $U_C(\infty)$ для цепи RC рассчитываются по следующим формулам:

\[
	U_C(0+) = E(0-) = -2 \, V
\]

\[
	I(0+) = \frac{E + U_c}{R} = \frac{2\, V + 2\, V}{40 \, \Omega} = 100 \, mA
\]

\[
	I(\infty) = I(0-) = 0
\]

\[
	U_C(\infty) = E(0+) = 2 \, V
\]

Значение времени переходного процесса \( t_{0.5} \) определяется как время, за которое ток достигает половины своего амплитудного значения. Постоянная времени \( \tau \) определяется как:

\[
	\tau = \frac{t_{0.5}}{\ln 2}
\]

\textbf{Вычисление постоянной времени \( \tau \):}

\[
	\tau = \frac{t_{0.5_I}}{\ln 2} = \frac{8.418554 \, \text{мс}}{0.69314718} = 12.144 \, \text{мс}
\]


Постоянная времени \( \tau \) была определена экспериментально и составляет
\( 12.144 \, \text{мс} \). Это значение будет использовано для расчёта соответствующих токов и напряжений в цепи.
