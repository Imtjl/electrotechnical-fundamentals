\documentclass[a4paper]{article}

% Packages
\usepackage[14pt]{extsizes}
\usepackage[T2A]{fontenc}
\usepackage[russian]{babel}
\usepackage[left=20mm, top=15mm, right=15mm, bottom=20mm]{geometry}
\usepackage{graphicx} % For images
\usepackage{amsmath, amssymb} % For equations
\usepackage{booktabs} % For better tables
\usepackage{pgfplots} % For plotting graphs
\usepackage{xcolor} % For color support
\usepackage{caption} % For captioning tables and figures
\usepackage{float} % For precise float placement (images, tables)
\usepackage{array} % For better table management
\pgfplotsset{compat=1.17}

% Importing custom definitions (lstset, tikzset, etc.)
\input{../common/title-lab.tex}
\input{../common/title-hw.tex}
% \input{../common/lstset.tex}
% \input{../common/tikzset.tex}

\begin{document}

% Title page
\labtitle{1}{Исследование характеристик источника электрической энергии постоянного тока}{3331}{33}{Дворкин Борис Александрович}{11.09.2024}{18.09.2024}{09.10.2024}{}
\thispagestyle{empty}

\newpage
\pagestyle{plain}
\setcounter{page}{1}

% Section 1: Схема эксперимента
\section{Схема эксперимента}
На рисунке 1.1 представлена схема замещения источника электрической энергии постоянного тока и нагрузки, созданная в приложении LTspice.

\begin{figure}[H]
	\centering
	\includegraphics[width=0.6\textwidth]{schema.png} % Make sure the path to the image is correct
	\caption{Схема замещения источника электрической энергии в LTspice.}
\end{figure}

% Section 2: Таблица измерений
\section{Заполненная таблица 1.1}
\begin{table}[H]
	\centering
	\addtocounter{table}{0}
	\caption*{Таблица 1.1: Результаты измерений и расчётов}
	\begin{tabular}{|c|c|c|c|c|c|c|}
		\hline
		k  & \multicolumn{2}{c|}{Измерения} & \multicolumn{4}{c|}{Расчёт: r = 594.208 [Ом], E = 12 [B], Isc = 20 [мА]}                                            \\
		\hline
		0  & $R_n$ [Ом]                     & $U_n$ [В]                                                                & $I_n$ [мА] & $P_n$ [Вт] & $n$ & $r$ [Ом] \\
		\hline
		1  & $\infty$                       & 12.000                                                                   & 0.00       & 0.00       & 1.0 & --       \\
		2  & 5400                           & 10.692                                                                   & 1.98       & 0.021      & 0.9 & 600.00   \\
		3  & 2400                           & 9.504                                                                    & 3.96       & 0.038      & 0.8 & 600.00   \\
		4  & 1400                           & 8.316                                                                    & 5.94       & 0.049      & 0.7 & 600.00   \\
		5  & 900                            & 7.128                                                                    & 7.92       & 0.056      & 0.6 & 600.00   \\
		6  & 600                            & 5.940                                                                    & 9.9        & 0.059      & 0.5 & 600.00   \\
		7  & 400                            & 4.752                                                                    & 11.88      & 0.056      & 0.4 & 599.294  \\
		8  & 257                            & 3.563                                                                    & 13.864     & 0.049      & 0.3 & 600.709  \\
		9  & 150                            & 2.376                                                                    & 15.84      & 0.038      & 0.2 & 601.729  \\
		10 & 67                             & 1.193                                                                    & 17.806     & 0.021      & 0.1 & 543.756  \\
		11 & 0                              & 0.000                                                                    & 20         & 0.000      & 0.0 & --       \\
		\hline
	\end{tabular}
\end{table}

% Section 3: Пример расчёта
\newpage
\section{Пример расчёта для одной строки таблицы}

Для расчёта параметров используем следующие формулы:

\begin{itemize}
	\item Ток через нагрузку:
	      \[
		      I_n = \frac{U_n}{R_n}
	      \]

	\item Мощность, рассеиваемая на нагрузке:
	      \[
		      P_n = \frac{U_n^2}{R_n}
	      \]

	\item Коэффициент полезного действия:
	      \[
		      \eta_n = \frac{R_n}{R_n + r}
	      \]

	\item Внутреннее сопротивление источника:
	      \[
		      r_k = \frac{U_k - U_{k+1}}{I_{k+1} - I_k}
	      \]
\end{itemize}

Рассчитаем значения для строки \(n = 2\):

\begin{align*}
	I_2    & = \frac{U_2}{R_2} = \frac{10.692}{5400} = 1.98 \, \text{мА},                           \\
	P_2    & = \frac{U_2^2}{R_2} = \frac{10.692^2}{5400} \approx 0.021 \, \text{Вт},                \\
	\eta_2 & = \frac{R_2}{R_2 + r} = \frac{5400}{5400 + 600} = 0.9,                                 \\
	r_2    & = \frac{U_2 - U_3}{I_3 - I_2} = \frac{10.692 - 9.504}{3.96 - 1.98} = 600 \, \text{Ом}.
\end{align*}

Таким образом, для строки \(n = 2\) были рассчитаны следующие значения:
\[
	I_2 = 1.98 \, \text{мА}, \quad P_2 = 0.021 \, \text{Вт}, \quad \eta_2 = 0.9, \quad r_2 = 600 \, \text{Ом}.
\]


% Section 4: Расчётная внешняя характеристика источника
\include{external_charasteristic_graph}

% Section 5: Графики зависимости Pn(In) и η(In)
\input{power_graph}
\input{efficiency_graph}

% Section 6: Выводы по работе
\section{Выводы по работе}

В ходе данной лабораторной работы я исследовал внешнюю характеристику источника электрической 
энергии постоянного тока и определил параметры схемы его замещения на основе экспериментальных 
данных. Схема была собрана в программном обеспечении для моделирования аналоговых электронных 
схем «LTspice», где я измерил напряжение на резисторах при различных сопротивлениях нагрузки 
и выявил, что с уменьшением сопротивления резисторов напряжение на них падает, 
а ток в цепи увеличивается. Это изменение существенно влияет на распределение мощности в 
нагрузке и на эффективность работы источника.

В процессе работы я применил закон Джоуля-Ленца, 
который объясняет потери мощности на внутреннем сопротивлении источника при протекании тока. 
Согласно закону, выделяемая энергия на внутреннем сопротивлении источника определяется 
выражением \(Q = I^2 R t\). Увеличение тока приводит к большему выделению тепла внутри источника,
что снижает количество энергии, передаваемой на нагрузку, и соответственно уменьшает КПД.

Я провёл измерение напряжения холостого хода \( U_0 \), 
которое использовалось для расчёта тока короткого замыкания 
и определения внешней характеристики источника. 
Также я рассчитал токи, мощности и КПД на основе измерений, 
что позволило оценить внутреннее сопротивление источника. 
Все расчёты были проведены в Excel, где я составил таблицы с данными, а графики зависимостей 
\( P_n(I_n) \) и \( \eta(I_n) \) были построены в LaTeX. 
Это позволило оформить результаты в соответствии с научными стандартами и 
закрепить навыки работы с математическими пакетами для дальнейшей работы, 
включая подготовку к диплому.

В итоге, я пришёл к выводу, что внутреннее сопротивление источника играет критическую роль 
в его характеристиках: при высоких токах оно значительно снижает эффективность передачи 
энергии на нагрузку, что приводит к уменьшению КПД и снижению мощности. 
Эксперимент подтвердил теоретические модели, а полученные данные дали возможность точно 
рассчитать параметры реальных электрических цепей.

\end{document}
