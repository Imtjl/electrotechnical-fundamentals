\subsection{Введение}
В данной части лабораторной работы произведены измерения действующих значений входного напряжения, тока и фазового сдвига между ними для девяти различных двухполюсников, а также произведены сравнения результатов с расчётными значениями.

\subsection{Параметры элементов исследуемых схем}
\begin{enumerate}
	\item Расчёт амплитуды синусоидального напряжения:
	      \[
		      U_{max} =  U_{\text{д}} \cdot \sqrt{2} = 6 \cdot \sqrt{2} = 8.485 \, \text{В}
	      \]

	\item Известные значения:
	      \[
		      \begin{gathered}
			      U_{\text{д}} = 6 \, \text{В}, \psi_{\text{н}} = 0^{\circ}, f = 19.894 \, \text{Гц}, R_1 = 30 \, \text{Ом}, R_k = 5 \, \text{Ом} \\
			      L_k = 23.094 \, \text{мГн}, C = 71.454 \, \text{мкФ}
		      \end{gathered}
	      \]

\end{enumerate}

\subsection{Общие расчёты}
\begin{enumerate}
	\item Угловая частота:
	      \[
		      \omega = 2 \pi f = 2 \cdot 3.1416 \cdot 19.894 \approx 125 \, \text{рад/с} \\
	      \]

	\item Реактивная составляющая сопротивления ёмкостного элемента:
	      \[
		      X_c = \frac{1}{\omega C} = \frac{1}{125 \cdot 71.454 \cdot 10^{-6}} = 111.96 \, \text{Ом}
	      \]

	\item Реактивная составляющая сопротивления индуктивного элемента:
	      \[
		      X_L = \omega L = 125 \cdot 23.094 \cdot 10^{-3} = 2.887 \, \text{Ом}
	      \]

	\item Реактивная проводимость ёмкостного элемента:
	      \[
		      B_c = \omega C = 125 \cdot 71.454 \cdot 10^{-6} = 0.00893 \, \text{См}
	      \]

	\item Реактивная проводимость индуктивного элемента:
	      \[
		      B_k = \frac{X_L}{R_k^2 + X_L^2} = \frac{2.887}{5^2 + (2.887)^2} = 0.0866 \, \text{См}
	      \]
\end{enumerate}


\subsection{Двухполюсник 1}
\subsubsection{Схема исследуемой цепи}
\begin{figure}[H]
	\centering
	\includegraphics[width=1\textwidth]{./data/schema1}
	\caption{Схема замещения Двухполюсника 1 в LTspice.}
\end{figure}
\subsubsection{Расчётные формулы и расчёты}
\begin{enumerate}
	\item Определение топологии цепи:
	      \begin{align*}
		      p^*           & = 6 \, (\text{общее количество ветвей}),                                     \\
		      p_{\text{ит}} & = 1 \, (\text{количество ветвей с источниками тока}),                        \\
		      p             & = p^* - p_{\text{ит}} = 6 - 1 = 5 \, (\text{количество неизвестных токов}),  \\
		      q             & = 4 \, (\text{количество узлов}),                                            \\
		      n             & = p - (q - 1) = 5 - (4 - 1) = 2 \, (\text{количество независимых контуров}), \\
		      m_I           & = q - 1 = 4 - 1 = 3 \, (\text{количество уравнений по ЗКИ}),                 \\
		      m_{II}        & = n = 2 \, (\text{количество уравнений по ЗКП}).
	      \end{align*}

	\item Произвольно обозначим $p$ неизвестных токов, $q$ узлов и $n$ независимых контуров. В любом месте ветви обозначим стрелки и имена искомых токов. В узлах поставим порядковый номер (арабская цифра), обведённый окружностью. Для выбранных контуров укажем направление их обхода, и внутри контура укажем порядковый номер (римская цифра), обведённый окружностью.
	      \begin{figure}[H]
	\centering
	\begin{circuitikz}[american, scale=1.15]

		\draw
		(0,0)
		to[R, l=$R_3$, i=$I_3$] (10,0)
		-- (10, -3)
		to[short, *-] (10, -4.1);

		\draw
		(10, -3)
		to[R, l=$R_5$, i=$I_5$] (7.5, -3)
		to[I, l=$E_5$] (5, -3)
		to[short, *-] (5, -3)
		to[R, l=$R_2$, i=$I_2$] (2.5, -3)
		to[I, l=$E_2$] (0, -3);

		\draw
		(0,0)
		-- (0,-3)
		to[short, *-] (0, -3)
		to[R, l=$R_1$, i=$I_1$] (0, -6)
		-- (5, -6) to[short, *-] (10, -6)
		-- (10, -4.9);

		\draw
		(5, -3)
		to[R, l=$R_4$, i=$I_4$] (5, -6);

		\draw
		(10,-4.5) circle(0.4cm)
		(10,-4.45) edge[thick, -{Straight Barb[length=2mm]}] (10, -4.44)
		(10,-4.35) edge[thick, -{Straight Barb[length=2mm]}] (10, -4.34)
		node at (10.75, -4.5) {$J_6$};

		\draw[
			color=red!60, thick, dashed,
			postaction={decorate,decoration={
							markings,
							mark=at position 0.25 with {\arrow[scale=1.5,fill=red!90]{<}},
							mark=at position 0.5 with {\arrow[scale=1.5,fill=red!90]{<}},
							mark=at position 0.75 with {\arrow[scale=1.5,fill=red!90]{<}},
							mark=at position 1 with {\arrow[scale=1.5,fill=red!90]{<}},
						}}
		]
		(5, -1.5) ellipse [x radius=4cm, y radius=0.97cm]
		node[scale=1.2, draw, thick, solid, circle] {$II$};

		\draw[
			color=red!60, thick, dashed,
			postaction={decorate,decoration={
							markings,
							mark=at position 0.25 with {\arrow[scale=1.5,fill=red!90]{>}},
							mark=at position 0.5 with {\arrow[scale=1.5,fill=red!90]{>}},
							mark=at position 0.75 with {\arrow[scale=1.5,fill=red!90]{>}},
							mark=at position 1 with {\arrow[scale=1.5,fill=red!90]{>}},
						}}
		]
		(2.7, -4.7) ellipse [x radius=1.5cm, y radius=1cm]
		node[scale=1.2, draw, thick, solid, circle] {$I$};

		\draw (0, -3) node[draw, circle, fill=blue!10, scale=0.8] {\textcolor{blue}{1}};
		\draw (5, -3) node[draw, circle, fill=blue!10, scale=0.8] {\textcolor{blue}{2}};
		\draw (10, -3) node[draw, circle, fill=blue!10, scale=0.8] {\textcolor{blue}{3}};
		\draw (5, -6) node[draw, circle, fill=blue!10, scale=0.8] {\textcolor{blue}{4}};

	\end{circuitikz}
	\caption{Схема электрической цепи с контурами, узлами и токами}
\end{figure}


	\item Составим систему из $m_I$ уравнений по ЗКI и $m_{II}$ уравнений по ЗКII:
	      \[
		      \begin{cases}
			      \text{ЗК1:} & \begin{cases}
				                    I_2 - I_1 - I_3 = 0, \quad \text{для узла 1}, \\
				                    I_5 - I_4 - I_2 = 0, \quad \text{для узла 2}, \\
				                    I_3 - I_5 = -J_6, \quad \text{для узла 3},
			                    \end{cases}                        \\[10pt]
			      \text{ЗК2:} & \begin{cases}
				                    I_2 R_2 + I_1 R_1 - I_4 R_4 = E_2, \quad \text{для контура I}, \\
				                    I_3 R_3 + I_5 R_5 + I_2 R_2 = E_5 + E_2, \quad \text{для контура II}.
			                    \end{cases}
		      \end{cases}
	      \]

	\item Представим в матричной форме \( A \cdot X = F \):
	      \[
		      \begin{bmatrix}
			      -1  & 1   & -1  & 0    & 0   \\
			      0   & -1  & 0   & -1   & 1   \\
			      0   & 0   & 1   & 0    & -1  \\
			      R_1 & R_2 & 0   & -R_4 & 0   \\
			      0   & R_2 & R_3 & 0    & R_5
		      \end{bmatrix}
		      \cdot
		      \begin{bmatrix}
			      I_1 \\
			      I_2 \\
			      I_3 \\
			      I_4 \\
			      I_5
		      \end{bmatrix}
		      =
		      \begin{bmatrix}
			      0    \\
			      0    \\
			      -J_6 \\
			      E_2  \\
			      E_5 + E_2
		      \end{bmatrix}
	      \]

	\item Подставим численные значения:
	      \[
		      \begin{bmatrix}
			      -1 & 1  & -1 & 0  & 0  \\
			      0  & -1 & 0  & -1 & 1  \\
			      0  & 0  & 1  & 0  & -1 \\
			      8  & 6  & 0  & -2 & 0  \\
			      0  & 6  & 7  & 0  & 7
		      \end{bmatrix}
		      \cdot
		      \begin{bmatrix}
			      I_1 \\
			      I_2 \\
			      I_3 \\
			      I_4 \\
			      I_5
		      \end{bmatrix}
		      =
		      \begin{bmatrix}
			      0     \\
			      0     \\
			      -1,95 \\
			      34,5  \\
			      41,5
		      \end{bmatrix}
	      \]
	\item Решим систему уравнений:
	      \[
		      X
		      =
		      \begin{pmatrix}
			      I_1 \\
			      I_2 \\
			      I_3 \\
			      I_4 \\
			      I_5
		      \end{pmatrix}
		      =
		      A^{-1} \cdot F =
		      \begin{pmatrix}
			      2.116  \\
			      2.874  \\
			      0.758  \\
			      -0.166 \\
			      2.708
		      \end{pmatrix}
	      \]
\end{enumerate}

\subsubsection{Вектора входного напряжения и тока}
\begin{figure}[H]
	\centering
	\begin{tikzpicture}[scale=2.0]

		% Draw the grid
		\draw[very thin, gray] (-1.2,-1.2) grid (2.2,2.2);

		% Draw the axes
		\draw[->] (-1.2,0) -- (2.2,0) node[right] {$+1$};
		\draw[->] (0,-1.2) -- (0,2.2) node[above] {$+j$};

		% Draw the first vector (red) for current (I)
		\draw[->, red, thick] (0,0) -- (0.249,0) node[end, below] {$I = 0.249 \, \text{А}$};

		% Draw the second vector (blue) for voltage (U)
		\draw[->, blue, thick] (0,0) -- (0.6,0);
		\node[above right, blue] at (0.6, 0) {$U = 6 \, \text{В}$};

		% Draw the angle label
		\draw (0.1,0) arc[start angle=0, end angle=0, radius=1cm];
		\node[above right] at (-0.03,0) {\scriptsize $\phi = 0^{\circ}$};

	\end{tikzpicture}
\end{figure}


\subsection{Двухполюсник 2}
\subsubsection{Схема исследуемой цепи}
\begin{figure}[H]
	\centering
	\includegraphics[width=0.8\textwidth]{./data/schema2}
	\caption{Схема замещения Двухполюсника 2 в LTspice.}
\end{figure}
\subsubsection{Расчётные формулы и расчёты}
\begin{enumerate}
	\item Расчёт действующего тока в цепи:
	      \[
		      \begin{gathered}
			      I = \frac{U}{Z} = \frac{U}{\sqrt{R^2 + X^2}} \\
			      X = -X_C, R = 0 \implies I = \frac{U}{X_C} = \frac{6}{111.96} = 0.0536 \, \text{А}
		      \end{gathered}
	      \]
	\item Расчёт фазового сдвига:
	      \[
		      \phi = \arctan\left(-\inf\right) = -90^{\circ}
	      \]
\end{enumerate}

\subsubsection{Вектора входного напряжения и тока}
\begin{figure}[H]
	\centering
	\begin{tikzpicture}[scale=1.8]

		% Draw the grid
		\draw[very thin, gray] (-1.2,-1.2) grid (2.2,2.2);

		% Draw the axes
		\draw[->] (-1.2,0) -- (2.2,0) node[right] {$+1$};
		\draw[->] (0,-1.2) -- (0,2.2) node[above] {$+j$};

		% Draw the current vector (red) for current (I) with a phase shift of -90 degrees
		\draw[->, red, thick] (0,0) -- (0,0.536) node[end, right] {$I = 0.0536 \, \text{А}$};

		% Draw the voltage vector (blue) for voltage (U)
		\draw[->, blue, thick] (0,0) -- (0.6,0) node[end, below] {$U = 6 \, \text{В}$};

		% Draw the phase angle arc (from U to I)
		\draw[thick] (0.2,0) arc[start angle=0, end angle=90, radius=0.2];
		\node at (0.55,0.2) {\scriptsize $\phi = -90^\circ$};

		\node[below left] at (0,0) {$0$};

	\end{tikzpicture}
\end{figure}


\subsection{Двухполюсник 3}
\subsubsection{Схема исследуемой цепи}
\begin{figure}[H]
	\centering
	\includegraphics[width=0.5\textwidth]{./data/schema3}
	\caption{Схема замещения Двухполюсника 3 в LTspice.}
\end{figure}
\subsubsection{Расчётные формулы и расчёты}
\begin{enumerate}
	\item Применим типивые эквивалентные преобразования к \textbf{исходной схеме}:
	      \newcommand{\given}[1]{%
	\begin{figure}[H]
		\centering
		\begin{circuitikz}[american, scale=#1]

			\draw
			(0,0)
			to[R, l=$R_3$] (10,0)
			-- (10, -3)
			to[short, *-] (10, -4.1);

			\draw
			(10, -3)
			to[R, l=$R_5$] (7.5, -3)
			to[I, l=$E_5$] (5, -3)
			to[short, *-] (5, -3)
			to[R, l=$R_2$] (2.5, -3)
			to[I, l=$E_2$] (0, -3);

			\draw
			(0,0)
			-- (0,-3)
			to[short, *-] (0, -3)
			to[R, l=$R_1$] (0, -6)
			-- (5, -6) to[short, *-] (10, -6)
			-- (10, -4.9);

			\draw
			(5, -3)
			to[R, l=$R_4$] (5, -6);


			\draw
			(10,-4.5) circle(0.4cm)
			(10,-4.45) edge[thick, -{Straight Barb[length=2mm]}] (10, -4.44)
			(10,-4.35) edge[thick, -{Straight Barb[length=2mm]}] (10, -4.34)
			node at (10.75, -4.5) {$J_6$};

		\end{circuitikz}
		\caption{Исходная электрическая схема с обозначениями элементов}
	\end{figure}
}

	      \given{1}
	\item Расщепляем $J_6$ на $R_4, R_5, E_5$:
	      \begin{figure}[H]
	\centering
	\begin{circuitikz}[american, scale=1]

		\draw
		(0,0)
		to[R, l=$R_3$] (10,0)
		-- (10, -2)
		to[short, *-] (10, -2);

		\draw
		(10, -2)
		to[R, l_=$R_5$] (7, -2)
		to[I, l=$E_5$] (5, -2)
		to[short, *-] (5, -2)
		to[R, l=$R_2$] (2.5, -2)
		to[I, l=$E_2$] (0, -2);

		\draw
		(0,0)
		-- (0,-2)
		to[short, *-] (0, -3)
		to[R, l=$R_1$] (0, -6)
		-- (5, -6) to[short, *-] (5, -6);

		\draw
		(5, -2)
		-- (5, -3)
		to[R, l_=$R_4$] (5, -6);


		\draw
		(9.5, -2)
		to[short, *-] (9.5, -2)
		-- (9.5, -3)
		-- (8.85, -3);

		\draw
		(7.5, -2)
		to[short, *-] (7.5, -2)
		-- (7.5, -3)
		-- (8.05, -3);

		\draw[rotate around={-90:(8.5,-3)}]
		(8.5,-3.05) circle(0.4cm)
		(8.5,-2.95) edge[thick, -{Straight Barb[length=2mm]}] (8.5, -2.94)
		(8.5,-2.85) edge[thick, -{Straight Barb[length=2mm]}] (8.5, -2.84)
		node at (9.25, -3) {$J_{\text{э2}}$};

		\draw
		(5, -3.5)
		to[short, *-] (5, -3.5)
		-- (6, -3.5)
		-- (6, -4.1);

		\draw
		(5, -5.5)
		to[short, *-] (5, -5.5)
		-- (6, -5.5)
		-- (6, -4.9);

		\draw
		(6,-4.5) circle(0.4cm)
		(6,-4.45) edge[thick, -{Straight Barb[length=2mm]}] (6, -4.44)
		(6,-4.35) edge[thick, -{Straight Barb[length=2mm]}] (6, -4.34)
		node at (6.75, -4.5) {$J_{\text{э1}}$};


	\end{circuitikz}
\end{figure}

	\item $J_{\text{э}1}$ \parallel $R_4 \rightarrow E_4$, $J_{\text{э}2}$ \parallel $R_5 \rightarrow E_{5'}$:
	      \begin{figure}[H]
	\centering
	\begin{circuitikz}[american, scale=1]

		\draw
		(0,0)
		to[R, l=$R_3$] (10,0)
		-- (10, -2)
		to[short, *-] (10, -2);

		\draw
		(10, -2)
		to[R, l=$R_5$] (8, -2);

		\draw
		(7, -2) to[I, l_=$E_{5'}$] (8, -2);

		\draw
		(7, -2)
		to[I, l=$E_5$] (5, -2)
		to[short, *-] (5, -2)
		to[R, l=$R_2$] (2.5, -2)
		to[I, l=$E_2$] (0, -2);

		\draw
		(0,0)
		-- (0,-2)
		to[short, *-] (0, -5)
		to[R, l=$R_{14}$] (3, -5)
		to[I, l=$E_4$] (5, -5)
		-- (5, -2);
	\end{circuitikz}
\end{figure}


	      \[
		      \begin{gathered}
			      E_4 = R_4 \cdot J_{\text{э1}} = 2 \cdot 1.95 = 3.9 \, \text{В} \\
			      E_{5'} = R_5 \cdot J_{\text{э2}} = 7 \cdot 1.95 = 13.65 \, \text{В} \\
			      R_{35} = R_3 + R_5 = 14 \, \Omega \\
			      E_{55'} = E_5 - E_{5'} = 6.65 \, \text{В}
		      \end{gathered}
	      \]
	\item $R_{14}, E_4$ \parallel $R_{35}, E_{55'}$:
	      \begin{figure}[H]
	\centering
	\begin{circuitikz}[american, scale=1]

		\draw
		(0,0)
		-- (10, 0)
		-- (10, -2)
		to[R, l=$R_{1435}$] (8, -2)
		to[I, l=$E_{455'}$] (5, -2)
		to[R, l=$R_2$] (2.5, -2)
		to[I, l=$E_2$] (0, -2)
		-- (0, 0);

	\end{circuitikz}
\end{figure}


	      \[
		      \begin{gathered}
			      R_{1435} = \frac{1}{\frac{1}{R_{14}} + \frac{1}{R_{35}}} = \frac{1}{\frac{1}{10} + \frac{1}{14}} = 5.833 \, \Omega \\
			      E_{455'} = R_{1435} \cdot \left(\frac{E_4}{R_{14}} - \frac{E_{55'}}{R_{35}}\right) = 5.833 \cdot \left(\frac{3.9}{10} - \frac{6.65}{14}\right) = -0.496 \, \text{В} \\
		      \end{gathered}
	      \]
	\item Схема сведена к одноконтурной относительно ветви с искомым током. Искомый ток $I2$ определим с использованием ЗКII:
	      \[
		      \begin{gathered}
			      I_2 \cdot (R_2 + R_{1345}) = E_2 + E_{455'} \Leftrightarrow I_2 = \frac{E_2 + E_{455'}}{R_2 + R_{1345}} \\
			      I_2 = \frac{34.5 - 0.496}{6+5.833} = 2.874 \, \text{А}
		      \end{gathered}
	      \]

\end{enumerate}

\subsubsection{Вектора входного напряжения и тока}
\[
	\begin{gathered}
		I_x = I \cos(\phi), I_y = I sin(\phi) \\
		I_x = 0.0518 \cdot \cos(75^\circ) = 0.0134 \, \text{А}, \quad
		I_y = 0.0518 \cdot \sin(75^\circ) = 0.05 \, \text{А}
	\end{gathered}
\]
\begin{figure}[H]
	\centering
	\begin{tikzpicture}[scale=4.0]

		% Draw the grid
		\draw[very thin, gray] (-1.2,-1.2) grid (2.2,2.2);

		% Draw the axes
		\draw[->] (-1.2,0) -- (2.2,0) node[right] {$+1$};
		\draw[->] (0,-1.2) -- (0,2.2) node[above] {$+j$};

		% Draw the current vector (red) with a phase shift of -75 degrees
		\draw[->, red, thick] (0,0) -- ({0.518*cos(75)}, {0.518*sin(75)})
		node[end, right] {$I = 0.0518 \, \text{А}$};
		\draw[gray, thin, dashed] ({0.518*cos(75)},0) node[start, below, red] {\scriptsize0.0134} -- ({0.518*cos(75)}, {0.518*sin(75)});
		\draw[gray, thin, dashed] (0,{0.518*sin(75)}) node[start, left, red] {\scriptsize0.05} -- ({0.518*cos(75)}, {0.518*sin(75)});


		% Draw the voltage vector (blue) for voltage (U)
		\draw[->, blue, thick] (0,0) -- (1.8,0) node[end, above] {$U = 6 \, \text{В}$};

		% Draw the phase angle arc (from U to I)
		\draw[thick] (0.18,0) arc[start angle=0, end angle=75, radius=0.18];
		\node at (0.36,0.18) {\small $\phi = -75^\circ$};

		% Labels for the axis
		\node[below left] at (0,0) {$0$};

	\end{tikzpicture}
\end{figure}


\subsection{Двухполюсник 4}
\subsubsection{Схема исследуемой цепи}
\begin{figure}[H]
	\centering
	\includegraphics[width=1\textwidth]{./data/schema4}
	\caption{Схема замещения Двухполюсника 4 в LTspice.}
\end{figure}
\subsubsection{Расчётные формулы и расчёты}
\begin{enumerate}
	\item Коэффициент затухания:

	      \[
		      \delta = \frac{R}{2L} = \frac{20}{2 \cdot 0,48} = 20,833 \, \text{с}^{-1}
	      \]

	\item Резонансная частота:

	      \[
		      \omega_c = \sqrt{\frac{1}{LC} - \delta^2} = \sqrt{\frac{1}{0,48 \cdot 300 \cdot 10^{-6}} - \frac{125^2}{6^2}} \approx 80,687 \, \text{с}^{-1}
	      \]

	\item Эксперементальное определение коэффициента затухания:

	      \[
		      \delta^* = \frac{\ln{\left(\frac{I_{m1}}{I_{m2}}\right)}}{T} = \frac{\ln{\left(\frac{0,071082}{0,031584}\right)}}{0,0781} = 10,386 \, \text{с}^{-1}
	      \]

	\item Эксперементальное определение резонансной частоты:

	    \[
	        \omega_c^* = \frac{2\pi}{T} = \frac{2\pi}{0,0781} = 80,451 \, \text{с}^{-1}
	    \]
\end{enumerate}

\subsubsection{Вектора входного напряжения и тока}
\[
	\begin{gathered}
		I_x = I \cos(\phi), I_y = I sin(\phi) \\
		I_x = 1.039 \cdot \cos(-30.7^\circ) = 0.893 \, \text{А}, \quad
		I_y = 1.039 \cdot \sin(-30.7^\circ) = -0.531 \, \text{А}
	\end{gathered}
\]
\begin{figure}[H]
	\centering
	\begin{tikzpicture}[scale=5.0]

		% Draw the grid
		\draw[very thin, gray] (-1.2,-1.2) grid (2.2,2.2);

		% Draw the axes
		\draw[->] (-1.2,0) -- (2.2,0) node[right] {$+1$};
		\draw[->] (0,-1.2) -- (0,2.2) node[above] {$+j$};

		% Draw the current vector (red) with a phase shift of -75 degrees
		\draw[->, red, thick] (0,0) -- ({1.039*cos(-30.7)}, {1.039*sin(-30.7)})
		node[end, right] {$I = 1.039 \, \text{А}$};
		\draw[gray, thin, dashed] ({1.039*cos(-30.7)},0) node[start, below, red] {\scriptsize0.893} -- ({1.039*cos(-30.7)}, {1.039*sin(-30.7)});
		\draw[gray, thin, dashed] (0,{1.039*sin(-30.7)}) node[start, left, red] {\scriptsize0.531} -- ({1.039*cos(-30.7)}, {1.039*sin(-30.7)});


		% Draw the voltage vector (blue) for voltage (U)
		\draw[->, blue, thick] (0,0) -- (0.6,0) node[end, above] {$U = 6 \, \text{В}$};

		% Draw the phase angle arc (from U to I)
		\draw[thick] (0.18,0) arc[start angle=0, end angle=-30.7, radius=0.18];
		\node at (0.4,-0.07) {\scriptsize $\phi = 30.7^\circ$};

		% Labels for the axis
		\node[below left] at (0,0) {$0$};

	\end{tikzpicture}
\end{figure}


\subsection{Двухполюсник 5}
\subsubsection{Схема исследуемой цепи}
\begin{figure}[H]
	\centering
	\includegraphics[width=1\textwidth]{./data/schema5}
	\caption{Схема замещения Двухполюсника 5 в LTspice.}
\end{figure}
\subsubsection{Расчётные формулы и расчёты}
\begin{enumerate}
	\item Расчёт действующего тока в цепи:
	      \[
		      \begin{gathered}
			      I = \frac{U}{Z} = \frac{U}{\sqrt{R^2 + X^2}} \\
			      X = X_L, R = R_1 + R_k \implies I = \frac{U}{\sqrt{(R_1+R_k)^2+X_L^2}} = \frac{6}{\sqrt{(30+5)^2+2.887^2}} \\
			      \\
			      = 0.171 \, \text{А}
		      \end{gathered}
	      \]
	\item Расчёт фазового сдвига:
	      \[
		      \phi = \arctan\left(\frac{X_L}{R_1+R_k}\right) = \arctan\left(\frac{2.887}{30+5}\right) = 4.72^{\circ}
	      \]
\end{enumerate}

\subsubsection{Вектора входного напряжения и тока}
\[
	\begin{gathered}
		I_x = I \cos(\phi), I_y = I sin(\phi) \\
		I_x = 0.171 \cdot \cos(-4.72^\circ) = 0.17 \, \text{А}, \quad
		I_y = 0.171 \cdot \sin(-4.72^\circ) = -0.014 \, \text{А}
	\end{gathered}
\]
\begin{figure}[H]
	\centering
	\begin{tikzpicture}[scale=25.0]

		% Draw the grid
		\draw[very thin, gray] (-0.05,-0.3) grid (0.65,0.5);

		% Draw the axes
		\draw[->] (-0.05,0) -- (0.65,0) node[right] {$+1$};
		\draw[->] (0,-0.4) -- (0,0.5) node[above] {$+j$};

		% Draw the current vector (red) with a phase shift of -75 degrees
		\draw[->, red, thick] (0,0) -- ({0.171*cos(-4.72)}, {0.171*sin(-4.72)})
		node[end, right] {$I = 0.171 \, \text{А}$};
		\draw[gray, thin, dashed] ({0.171*cos(-4.72)},0) node[start, above, red] {\scriptsize0.17} -- ({0.171*cos(-4.72)}, {0.171*sin(-4.72)});
		\draw[gray, thin, dashed] (0,{0.171*sin(-4.72)}) node[start, left, red] {\scriptsize-0.014} -- ({0.171*cos(-4.72)}, {0.171*sin(-4.72)});


		% Draw the voltage vector (blue) for voltage (U)
		\draw[->, blue, thick] (0,0) -- (0.6,0) node[end, above] {$U = 6 \, \text{В}$};

		% Draw the phase angle arc (from U to I)
		\draw[thick] (0.11,0) arc[start angle=0, end angle=-4.72, radius=0.11];
		\node at (0.145,-0.006) {\scriptsize $\phi = 4.7^\circ$};

		% Labels for the axis
		\node[below left] at (0.005,0.005) {\scriptsize$0$};

	\end{tikzpicture}
\end{figure}


\subsection{Двухполюсник 6}
\subsubsection{Схема исследуемой цепи}
\begin{figure}[H]
	\centering
	\includegraphics[width=1\textwidth]{./data/schema6}
	\caption{Схема замещения Двухполюсника 6 в LTspice.}
\end{figure}
\subsubsection{Расчётные формулы и расчёты}
\begin{enumerate}
	\item Расчёт действующего тока в цепи:
	      \[
		      \begin{gathered}
			      I = \frac{U}{Z} = \frac{U}{\sqrt{R^2 + X^2}} \\
			      X = X_L-X_C, R = R_1 + R_k \implies I = \frac{U}{\sqrt{(R_1+R_k)^2+(X_L-X_C)^2}} = \\
			      = \frac{6}{\sqrt{(30+5)^2+(2.887-111.96)^2}} = 0.0524 \, \text{А}
		      \end{gathered}
	      \]
	\item Расчёт фазового сдвига:
	      \[
		      \phi = \arctan\left(\frac{X_L-X_C}{R_1+R_k}\right) = \arctan\left(\frac{2.887-111.96}{30+5}\right) = -72.209^{\circ}
	      \]
\end{enumerate}

\subsubsection{Вектора входного напряжения и тока}

\[
	\begin{gathered}
		I_x = I \cos(\phi), I_y = I sin(\phi) \\
		I_x = 0.0524 \cdot \cos(72.209^\circ) = 0.016 \, \text{А}, \quad
		I_y = 0.0524 \cdot \sin(72.209^\circ) = 0.05 \, \text{А}
	\end{gathered}
\]
\begin{figure}[H]
	\centering
	\begin{tikzpicture}[scale=27.0]

		% Draw the grid
		\draw[very thin, gray] (-0.04,-0.2) grid (0.63,0.3);

		% Draw the axes
		\draw[->] (-0.04,0) -- (0.61,0) node[right] {$+1$};
		\draw[->] (0,-0.2) -- (0,0.3) node[above] {$+j$};

		% Draw the current vector (red) with a phase shift of -75 degrees
		\draw[->, red, thick] (0,0) -- ({0.0524*cos(-72.209)}, {0.0524*sin(-72.209)})
		node[end, right] {$I = 0.0524 \, \text{А}$};
		\draw[gray, thin, dashed] ({0.0524*cos(-72.209)},0) node[start, above, red] {\scriptsize0.016} -- ({0.0524*cos(-72.209)}, {0.0524*sin(-72.209)});
		\draw[gray, thin, dashed] (0,{0.0524*sin(-72.209)}) node[start, left, red] {\scriptsize0.05} -- ({0.0524*cos(-72.209)}, {0.0524*sin(-72.209)});


		% Draw the voltage vector (blue) for voltage (U)
		\draw[->, blue, thick] (0,0) -- (0.58,0) node[end, above] {$U = 6 \, \text{В}$};

		% Draw the phase angle arc (from U to I)
		\draw[thick] (0.03,0) arc[start angle=0, end angle=-72.209, radius=0.03];
		\node at (0.065,-0.025) {\small $\phi = 72.209^\circ$};

		% Labels for the axis
		\node[below left] at (0.005,0.005) {\scriptsize$0$};

	\end{tikzpicture}
\end{figure}


\subsection{Двухполюсник 7}
\subsubsection{Схема исследуемой цепи}
\begin{figure}[H]
	\centering
	\includegraphics[width=1\textwidth]{./data/schema7}
	\caption{Схема замещения Двухполюсника 7 в LTspice.}
\end{figure}
\subsubsection{Расчётные формулы и расчёты}
\begin{enumerate}
	\item Расчёт действующего тока в цепи:
	      \[
		      \begin{gathered}
			      I = U \cdot Y = U \cdot \sqrt{G^2 + B^2} \\
			      G = \frac{1}{R_1}, B = -B_C \implies I = U \cdot \sqrt{\frac{1}{R_1^2} + B_C^2} = 6 \cdot \sqrt{\frac{1}{30^2} + 0.00893^2} = 0.207 \, \text{А}
		      \end{gathered}
	      \]
	\item Расчёт фазового сдвига:
	      \[
		      \phi = \arctan\left(\frac{-B_C}{\frac{1}{R_1}}\right) = \arctan\left(\frac{-0.00893}{0.03}\right) = -16.577^{\circ}
	      \]
\end{enumerate}

\subsubsection{Вектора входного напряжения и тока}
\[
	\begin{gathered}
		I_x = I \cos(\phi), I_y = I sin(\phi) \\
		I_x = 0.207 \cdot \cos(16.577^\circ) = 0.198 \, \text{А}, \quad
		I_y = 0.207 \cdot \sin(16.577^\circ) = 0.059 \, \text{А}
	\end{gathered}
\]
\begin{figure}[H]
	\centering
	\begin{tikzpicture}[scale=27.0]

		% Draw the grid
		\draw[very thin, gray] (-0.04,-0.2) grid (0.63,0.3);

		% Draw the axes
		\draw[->] (-0.04,0) -- (0.61,0) node[right] {$+1$};
		\draw[->] (0,-0.2) -- (0,0.3) node[above] {$+j$};

		% Draw the current vector (red) with a phase shift of -75 degrees
		\draw[->, red, thick] (0,0) -- ({0.207*cos(16.577)}, {0.207*sin(16.577)})
		node[end, right] {$I = 0.207 \, \text{А}$};
		\draw[gray, thin, dashed] ({0.207*cos(16.577)},0) node[start, above, red] {\scriptsize0.198} -- ({0.207*cos(16.577)}, {0.207*sin(16.577)});
		\draw[gray, thin, dashed] (0,{0.207*sin(16.577)}) node[start, left, red] {\scriptsize0.059} -- ({0.207*cos(16.577)}, {0.207*sin(16.577)});


		% Draw the voltage vector (blue) for voltage (U)
		\draw[->, blue, thick] (0,0) -- (0.58,0) node[end, above] {$U = 6 \, \text{В}$};

		% Draw the phase angle arc (from U to I)
		\draw[thick] (0.03,0) arc[start angle=0, end angle=16.577, radius=0.03];
		\node at (0.065,0.007) {\scriptsize $\phi = 16.577^\circ$};

		% Labels for the axis
		\node[below left] at (0.005,0.005) {\scriptsize$0$};

	\end{tikzpicture}
\end{figure}


\subsection{Двухполюсник 8}
\subsubsection{Схема исследуемой цепи}
\begin{figure}[H]
	\centering
	\includegraphics[width=0.7\textwidth]{./data/schema8}
	\caption{Схема замещения Двухполюсника 8 в LTspice.}
\end{figure}
\subsubsection{Расчётные формулы и расчёты}
\begin{enumerate}
	\item Расчёт действующего тока в цепи:
	      \[
		      \begin{gathered}
			      I = U \cdot Y = U \cdot \sqrt{G^2 + B^2} \\
			      G = G_1+G_k, B = B_k-B_1 \implies I = U \cdot \sqrt{(G_1+G_k)^2+(B_k-B_1)^2} = \\
			      = U \cdot \sqrt{\left(\frac{1}{R_1}+\frac{R_k}{R_k^2+X_L^2}\right)^2+\left(\frac{X_L}{R_k^2+X_L^2}-0\right)^2} = \\
			      = 6 \cdot \sqrt{\left(\frac{1}{30}+\frac{5}{5^2+2.887^2}\right)^2+\left(\frac{2.887}{5^2+2.887^2}\right)^2} = 1.217 \, \text{А}
		      \end{gathered}
	      \]
	\item Расчёт фазового сдвига:
	      \[
		      \phi = \arctan\left(\frac{B_k-B_1}{G_1+G_k}\right) = \arctan\left(\frac{0.0866}{0.183}\right) = 25.325^{\circ}
	      \]
\end{enumerate}

\subsubsection{Вектора входного напряжения и тока}
\[
	\begin{gathered}
		I_x = I \cos(\phi), I_y = I sin(\phi) \\
		I_x = 1.217 \cdot \cos(-25.325^\circ) = 1.1 \, \text{А}, \quad
		I_y = 1.217 \cdot \sin(-25.325^\circ) = -0.521 \, \text{А}
	\end{gathered}
\]
\begin{figure}[H]
	\centering
	\begin{tikzpicture}[scale=3.5]

		% Draw the grid
		\draw[very thin, gray] (-1.2,-1.2) grid (2.2,2.2);

		% Draw the axes
		\draw[->] (-1.2,0) -- (2.2,0) node[right] {$+1$};
		\draw[->] (0,-1.2) -- (0,2.2) node[above] {$+j$};

		% Draw the current vector (red) with a phase shift of -75 degrees
		\draw[->, red, thick] (0,0) -- ({1.217*cos(-25.325)}, {1.217*sin(-25.325)})
		node[end, right] {$I = 1.217 \, \text{А}$};
		\draw[gray, thin, dashed] ({1.217*cos(-25.325)},0) node[start, above, red] {\small1.1} -- ({1.217*cos(-25.325)}, {1.217*sin(-25.325)});
		\draw[gray, thin, dashed] (0,{1.217*sin(-25.325)}) node[start, left, red] {\small-0.521} -- ({1.217*cos(-25.325)}, {1.217*sin(-25.325)});


		% Draw the voltage vector (blue) for voltage (U)
		\draw[->, blue, thick] (0,0) -- (0.6,0) node[end, above] {$U = 6 \, \text{В}$};

		% Draw the phase angle arc (from U to I)
		\draw[thick] (0.18,0) arc[start angle=0, end angle=-25.325, radius=0.18];
		\node at (0.5,-0.07) {\scriptsize $\phi = 25.325^\circ$};

		% Labels for the axis
		\node[below left] at (0,0) {$0$};

	\end{tikzpicture}
\end{figure}


\subsection{Двухполюсник 9}
\subsubsection{Схема исследуемой цепи}
\begin{figure}[H]
	\centering
	\includegraphics[width=1\textwidth]{./data/schema9}
	\caption{Схема замещения Двухполюсника 9 в LTspice.}
\end{figure}
\subsubsection{Расчётные формулы и расчёты}
\begin{enumerate}
	\item Расчёт действующего тока в цепи:
	      \[
		      \begin{gathered}
			      I = U \cdot Y = U \cdot \sqrt{G^2 + B^2} \\
			      G = G_1+G_k, B = B_k-B_1 \implies I = U \cdot \sqrt{(G_1+G_k)^2+(B_k-B_1)^2} = \\
			      = U \cdot \sqrt{\left(\frac{R_1}{R_1^2+X_C^2}+\frac{R_k}{R_k^2+X_L^2}\right)^2+\left(\frac{X_L}{R_k^2+X_L^2}-\frac{X_C}{R_1^2+X_C^2}\right)^2} = \\
			      = 6 \cdot \sqrt{\left(\frac{30}{30^2+111.96^2}+\frac{5}{5^2+2.887^2}\right)^2+\left(\frac{2.887}{5^2+2.887^2}-\frac{111.96}{30^2+111.96^2}\right)^2} = \\
			      = 1.027 \, \text{А}
		      \end{gathered}
	      \]
	\item Расчёт фазового сдвига:
	      \[
		      \phi = \arctan\left(\frac{B_k-B_1}{G_1+G_k}\right) = \arctan\left(\frac{0.0783}{0.152}\right) = 27.254^{\circ}
	      \]
\end{enumerate}

\subsubsection{Вектора входного напряжения и тока}
\[
	\begin{gathered}
		I_x = I \cos(\phi), I_y = I sin(\phi) \\
		I_x = 1.027 \cdot \cos(-27.254^\circ) = 0.913 \, \text{А}, \quad
		I_y = 1.027 \cdot \sin(-27.254^\circ) = -0.47 \, \text{А}
	\end{gathered}
\]
\begin{figure}[H]
	\centering
	\begin{tikzpicture}[scale=3.5]

		% Draw the grid
		\draw[very thin, gray] (-1.2,-1.2) grid (2.2,2.2);

		% Draw the axes
		\draw[->] (-1.2,0) -- (2.2,0) node[right] {$+1$};
		\draw[->] (0,-1.2) -- (0,2.2) node[above] {$+j$};

		% Draw the current vector (red) with a phase shift of -75 degrees
		\draw[->, red, thick] (0,0) -- ({1.027*cos(-27.254)}, {1.027*sin(-27.254)})
		node[end, right] {$I = 1.027 \, \text{А}$};
		\draw[gray, thin, dashed] ({1.027*cos(-27.254)},0) node[start, below, red] {\small0.913} -- ({1.027*cos(-27.254)}, {1.027*sin(-27.254)});
		\draw[gray, thin, dashed] (0,{1.027*sin(-27.254)}) node[start, left, red] {\small-0.47} -- ({1.027*cos(-27.254)}, {1.027*sin(-27.254)});


		% Draw the voltage vector (blue) for voltage (U)
		\draw[->, blue, thick] (0,0) -- (0.6,0) node[end, above] {$U = 6 \, \text{В}$};

		% Draw the phase angle arc (from U to I)
		\draw[thick] (0.18,0) arc[start angle=0, end angle=-27.254, radius=0.18];
		\node at (0.5,-0.07) {\scriptsize $\phi = 27.254^\circ$};

		% Labels for the axis
		\node[below left] at (0,0) {$0$};

	\end{tikzpicture}
\end{figure}


\subsection{Заполненная таблица 2.2}
Для каждого двухполюсника 1-9, представленного выше, были не только произведены теоретические расчёты действующего тока и фазового сдвига, но и произведено построение временных диаграмм, из которых величины действующего напряжения, тока и фазового сдвига определены эксперементально. Для напряжения и тока были измерены амплитудные значения и вычислены по формуле:

\[
	\begin{gathered}
		U_{\text{д}} = \frac{U_{max}}{\sqrt{2}} \\
		I_{\text{д}} = \frac{I_{max}}{\sqrt{2}} \\
	\end{gathered}
\]

А фазовый сдвиг рассчитан следующим образом:

\[
	\phi = 180^\circ \cdot \frac{\delta h}{h}
\]
, где $\delta h$ - расстояние между моментами перехода синусоид напряжения и тока от отрицательных значений к положительным, а $h$ - половина периода синусоиды, измеренная в секундах.

\begin{table}[H]
	\captionsetup{labelformat=empty}
	\centering
	\includegraphics[width=1\textwidth]{./data/table_2-2.png}
	\caption{Итоговая таблица 2.2}
\end{table}

\subsection{Выводы}
В результате выполнения первой части лабораторной работы я исследовал 9 двухполюсников и рассчитал их действующие значения входного тока и напряжения, а также определил фазовый сдвиг между этими величинами.

В ходе исследования я эксперементально подтвердил теоретические значения величин действующего тока, напряжения и фазового сдвига, что подтвердило корректность опытов. Исходя из этого можно корректно заключить, как синусоидальный ток влиял на различные двухполюсники.

А именно, в первом двухполюснике нулевой сдвиг по фазе, что указывает что ток и напряжение изменяются синхронно, что очевидно, т.к. в цепи только лишь один резистор, и это можно подтвердить теоретически, т.к. сдвиг по фазе это соотношение реактивного и активного сопротивлений, а в цепи с одним лишь резистором реативное сопротивление отсутствует.

В двухполюсниках 2,3,7 в дело вступает ёмкостной элемент. В прошлой лабораторной я исследовал переходные процессы в электрических цепях и сделал вывод о том, что ёмкостной элемент сопротивляется изменению напряжения, а индуктивный - изменению тока. Следовательно, в данных двухполюсниках ток будет \textbf{опережать} напряжение.

В двухполюсниках 4,5,8 аналогично, но с индуктивным элементом. Ток уже будет \textbf{запаздывать} относительно напряжения, т.к. индуктивный элемент будет сопротивляться его изменению.

А в двухполюсниках 6,9 происходит самое интересное - катушка и конденсатор совмещены в одной цепи, и каждый сопротивляется изменению тока и напряжения соответственно, смещая обе фазы и тем самым \textit{сближая} или наоборот \textit{отдаляя} их друг от друга. Это можно наблюдать по значениям сдвига по фазе для данных двухполюсников. В двухполюснике 7 не было индуктивного элемента, в 6 добавился - и отдалил фазу тока, добавив задержку. И наоборот в двухполюснике 5 не было ёмкостного элемента, и добавили сразу большой - поначалу фазы могли бы сблизиться в нулевой сдвиг, если бы ёмкость была меньше, но так как мы добавили сразу большую ёмкость, то фазы далее отдалились аж до -72^\circ.

Также стоит отметить, что погрешности эмпирически полученных значений фазового сдвига, входного тока и напряжения связаны исключительно с неточностью вычислений и накоплением погрешностью округления. Если снимать показания с большим разрешением, то величины сходятся 1:1.
