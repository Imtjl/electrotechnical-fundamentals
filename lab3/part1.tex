\subsection{Введение}
В данной части лабораторной работы произведены измерения действующих значений входного напряжения, тока и фазового сдвига между ними для девяти различных двухполюсников, а также произведены сравнения результатов с расчётными значениями.

\subsection{Параметры источника}

\subsection{Общие расчёты}
\begin{enumerate}
	\item Угловая частота:
	      \[
		      \omega = 2 \pi f = 2 \cdot 3.1416 \cdot 19.894 \approx 125 \, \text{рад/с} \\
	      \]

	\item Реактивная составляющая сопротивления ёмкостного элемента:
	      \[
		      X_c = \frac{1}{\omega C} = \frac{1}{125 \cdot 71.454 \cdot 10^{-6}} = 111.96 \, \text{Ом}
	      \]

	\item Реактивная составляющая сопротивления индуктивного элемента:
	      \[
		      X_L = \omega L = 125 \cdot 23.094 \cdot 10^{-3} = 2.887 \, \text{Ом}
	      \]

	\item Реактивная проводимость ёмкостного элемента:
	      \[
		      B_c = \omega C = 125 \cdot 71.454 \cdot 10^{-6} = 0.00893 \, \text{См}
	      \]

	\item Реактивная проводимость индуктивного элемента:
	      \[
		      B_k = \frac{X_L}{R_k^2 + X_L^2} = \frac{2.887}{5^2 + (2.887)^2} = 0.0866 \, \text{См}
	      \]
\end{enumerate}


\subsection{Двухполюсник 1}
\subsubsection{Схема исследуемой цепи}
dgfdgd
\subsubsection{Расчётные формулы и расчёты}
\subsubsection{Векторная диаграмма входного напряжения и тока}

\subsection{Двухполюсник 2}
\subsubsection{Схема исследуемой цепи}
\subsubsection{Расчётные формулы и расчёты}
\subsubsection{Векторная диаграмма входного напряжения и тока}

\subsection{Двухполюсник 3}
\subsubsection{Схема исследуемой цепи}
\subsubsection{Расчётные формулы и расчёты}
\subsubsection{Векторная диаграмма входного напряжения и тока}

\subsection{Двухполюсник 4}
\subsubsection{Схема исследуемой цепи}
\subsubsection{Расчётные формулы и расчёты}
\subsubsection{Векторная диаграмма входного напряжения и тока}

\subsection{Двухполюсник 5}
\subsubsection{Схема исследуемой цепи}
\subsubsection{Расчётные формулы и расчёты}
\subsubsection{Векторная диаграмма входного напряжения и тока}

\subsection{Двухполюсник 6}
\subsubsection{Схема исследуемой цепи}
\subsubsection{Расчётные формулы и расчёты}
\subsubsection{Векторная диаграмма входного напряжения и тока}

\subsection{Двухполюсник 7}
\subsubsection{Схема исследуемой цепи}
\subsubsection{Расчётные формулы и расчёты}
\subsubsection{Векторная диаграмма входного напряжения и тока}

\subsection{Двухполюсник 8}
\subsubsection{Схема исследуемой цепи}
\subsubsection{Расчётные формулы и расчёты}
\subsubsection{Векторная диаграмма входного напряжения и тока}

\subsection{Двухполюсник 9}
\subsubsection{Схема исследуемой цепи}
\subsubsection{Расчётные формулы и расчёты}
\subsubsection{Векторная диаграмма входного напряжения и тока}

\subsection{Заполненная таблица 2.2}

\subsection{Выводы}
