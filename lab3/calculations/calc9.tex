\begin{enumerate}
	\item Расчёт действующего тока в цепи:
	      \[
		      \begin{gathered}
			      I = U \cdot Y = U \cdot \sqrt{G^2 + B^2} \\
			      G = G_1+G_k, B = B_k-B_1 \implies I = U \cdot \sqrt{(G_1+G_k)^2+(B_k-B_1)^2} = \\
			      = U \cdot \sqrt{\left(\frac{R_1}{R_1^2+X_C^2}+\frac{R_k}{R_k^2+X_L^2}\right)^2+\left(\frac{X_L}{R_k^2+X_L^2}-\frac{X_C}{R_1^2+X_C^2}\right)^2} = \\
			      = 6 \cdot \sqrt{\left(\frac{30}{30^2+111.96^2}+\frac{5}{5^2+2.887^2}\right)^2+\left(\frac{2.887}{5^2+2.887^2}-\frac{111.96}{30^2+111.96^2}\right)^2} = \\
			      = 1.027 \, \text{А}
		      \end{gathered}
	      \]
	\item Расчёт фазового сдвига:
	      \[
		      \phi = \arctan\left(\frac{B_k-B_1}{G_1+G_k}\right) = \arctan\left(\frac{0.0783}{0.152}\right) = 27.254^{\circ}
	      \]
\end{enumerate}
