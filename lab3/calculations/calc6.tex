\begin{enumerate}
	\item Расчёт действующего тока в цепи:
	      \[
		      \begin{gathered}
			      I = \frac{U}{Z} = \frac{U}{\sqrt{R^2 + X^2}} \\
			      X = X_L-X_C, R = R_1 + R_k \implies I = \frac{U}{\sqrt{(R_1+R_k)^2+(X_L-X_C)^2}} = \\
			      = \frac{6}{\sqrt{(30+5)^2+(2.887-111.96)^2}} = 0.0524 \, \text{А}
		      \end{gathered}
	      \]
	\item Расчёт фазового сдвига:
	      \[
		      \phi = \arctan\left(\frac{X_L-X_C}{R_1+R_k}\right) = \arctan\left(\frac{2.887-111.96}{30+5}\right) = -72.209^{\circ}
	      \]
\end{enumerate}
