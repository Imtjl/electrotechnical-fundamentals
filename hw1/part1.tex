Расчёт значения всех неизвестных токов в представленной на рисунке 1 цепи с помощью Законов Кирхгофа.

\subsection{Найти}
Все неизвестные токи в цепи: \(I_1, I_2, I_3, I_4, I_5 = ?\) \\
(Используя только I и II законы Кирхгофа)

\subsection{Решение}
\begin{enumerate}
	\item Определение топологии цепи:
	      \begin{align*}
		      p^*           & = 6 \, (\text{общее количество ветвей}),                                     \\
		      p_{\text{ит}} & = 1 \, (\text{количество ветвей с источниками тока}),                        \\
		      p             & = p^* - p_{\text{ит}} = 6 - 1 = 5 \, (\text{количество неизвестных токов}),  \\
		      q             & = 4 \, (\text{количество узлов}),                                            \\
		      n             & = p - (q - 1) = 5 - (4 - 1) = 2 \, (\text{количество независимых контуров}), \\
		      m_I           & = q - 1 = 4 - 1 = 3 \, (\text{количество уравнений по ЗКИ}),                 \\
		      m_{II}        & = n = 2 \, (\text{количество уравнений по ЗКП}).
	      \end{align*}

	\item Произвольно обозначим $p$ неизвестных токов, $q$ узлов и $n$ независимых контуров. В любом месте ветви обозначим стрелки и имена искомых токов. В узлах поставим порядковый номер (арабская цифра), обведённый окружностью. Для выбранных контуров укажем направление их обхода, и внутри контура укажем порядковый номер (римская цифра), обведённый окружностью.
	      \begin{figure}[H]
	\centering
	\begin{circuitikz}[american, scale=1.15]

		\draw
		(0,0)
		to[R, l=$R_3$, i=$I_3$] (10,0)
		-- (10, -3)
		to[short, *-] (10, -4.1);

		\draw
		(10, -3)
		to[R, l=$R_5$, i=$I_5$] (7.5, -3)
		to[I, l=$E_5$] (5, -3)
		to[short, *-] (5, -3)
		to[R, l=$R_2$, i=$I_2$] (2.5, -3)
		to[I, l=$E_2$] (0, -3);

		\draw
		(0,0)
		-- (0,-3)
		to[short, *-] (0, -3)
		to[R, l=$R_1$, i=$I_1$] (0, -6)
		-- (5, -6) to[short, *-] (10, -6)
		-- (10, -4.9);

		\draw
		(5, -3)
		to[R, l=$R_4$, i=$I_4$] (5, -6);

		\draw
		(10,-4.5) circle(0.4cm)
		(10,-4.45) edge[thick, -{Straight Barb[length=2mm]}] (10, -4.44)
		(10,-4.35) edge[thick, -{Straight Barb[length=2mm]}] (10, -4.34)
		node at (10.75, -4.5) {$J_6$};

		\draw[
			color=red!60, thick, dashed,
			postaction={decorate,decoration={
							markings,
							mark=at position 0.25 with {\arrow[scale=1.5,fill=red!90]{<}},
							mark=at position 0.5 with {\arrow[scale=1.5,fill=red!90]{<}},
							mark=at position 0.75 with {\arrow[scale=1.5,fill=red!90]{<}},
							mark=at position 1 with {\arrow[scale=1.5,fill=red!90]{<}},
						}}
		]
		(5, -1.5) ellipse [x radius=4cm, y radius=0.97cm]
		node[scale=1.2, draw, thick, solid, circle] {$II$};

		\draw[
			color=red!60, thick, dashed,
			postaction={decorate,decoration={
							markings,
							mark=at position 0.25 with {\arrow[scale=1.5,fill=red!90]{>}},
							mark=at position 0.5 with {\arrow[scale=1.5,fill=red!90]{>}},
							mark=at position 0.75 with {\arrow[scale=1.5,fill=red!90]{>}},
							mark=at position 1 with {\arrow[scale=1.5,fill=red!90]{>}},
						}}
		]
		(2.7, -4.7) ellipse [x radius=1.5cm, y radius=1cm]
		node[scale=1.2, draw, thick, solid, circle] {$I$};

		\draw (0, -3) node[draw, circle, fill=blue!10, scale=0.8] {\textcolor{blue}{1}};
		\draw (5, -3) node[draw, circle, fill=blue!10, scale=0.8] {\textcolor{blue}{2}};
		\draw (10, -3) node[draw, circle, fill=blue!10, scale=0.8] {\textcolor{blue}{3}};
		\draw (5, -6) node[draw, circle, fill=blue!10, scale=0.8] {\textcolor{blue}{4}};

	\end{circuitikz}
	\caption{Схема электрической цепи с контурами, узлами и токами}
\end{figure}


	\item Составим систему из $m_I$ уравнений по ЗКI и $m_{II}$ уравнений по ЗКII:
	      \[
		      \begin{cases}
			      \text{ЗК1:} & \begin{cases}
				                    I_2 - I_1 - I_3 = 0, \quad \text{для узла 1}, \\
				                    I_5 - I_4 - I_2 = 0, \quad \text{для узла 2}, \\
				                    I_3 - I_5 = -J_6, \quad \text{для узла 3},
			                    \end{cases}                        \\[10pt]
			      \text{ЗК2:} & \begin{cases}
				                    I_2 R_2 + I_1 R_1 - I_4 R_4 = E_2, \quad \text{для контура I}, \\
				                    I_3 R_3 + I_5 R_5 + I_2 R_2 = E_5 + E_2, \quad \text{для контура II}.
			                    \end{cases}
		      \end{cases}
	      \]

	\item Представим в матричной форме \( A \cdot X = F \):
	      \[
		      \begin{bmatrix}
			      -1  & 1   & -1  & 0    & 0   \\
			      0   & -1  & 0   & -1   & 1   \\
			      0   & 0   & 1   & 0    & -1  \\
			      R_1 & R_2 & 0   & -R_4 & 0   \\
			      0   & R_2 & R_3 & 0    & R_5
		      \end{bmatrix}
		      \cdot
		      \begin{bmatrix}
			      I_1 \\
			      I_2 \\
			      I_3 \\
			      I_4 \\
			      I_5
		      \end{bmatrix}
		      =
		      \begin{bmatrix}
			      0    \\
			      0    \\
			      -J_6 \\
			      E_2  \\
			      E_5 + E_2
		      \end{bmatrix}
	      \]

	\item Подставим численные значения:
	      \[
		      \begin{bmatrix}
			      -1 & 1  & -1 & 0  & 0  \\
			      0  & -1 & 0  & -1 & 1  \\
			      0  & 0  & 1  & 0  & -1 \\
			      8  & 6  & 0  & -2 & 0  \\
			      0  & 6  & 7  & 0  & 7
		      \end{bmatrix}
		      \cdot
		      \begin{bmatrix}
			      I_1 \\
			      I_2 \\
			      I_3 \\
			      I_4 \\
			      I_5
		      \end{bmatrix}
		      =
		      \begin{bmatrix}
			      0     \\
			      0     \\
			      -1,95 \\
			      34,5  \\
			      41,5
		      \end{bmatrix}
	      \]
	\item Решим систему уравнений:
	      \[
		      X
		      =
		      \begin{pmatrix}
			      I_1 \\
			      I_2 \\
			      I_3 \\
			      I_4 \\
			      I_5
		      \end{pmatrix}
		      =
		      A^{-1} \cdot F =
		      \begin{pmatrix}
			      2.116  \\
			      2.874  \\
			      0.758  \\
			      -0.166 \\
			      2.708
		      \end{pmatrix}
	      \]
\end{enumerate}


\subsection{Ответ}
Рассчитанные значения неизвестных токов в цепи:

\[
I_1 = 2.116 \, \text{А}, \quad
I_2 = 2.874 \, \text{А}, \quad
I_3 = 0.758 \, \text{А}, \quad
I_4 = -0.166 \, \text{А}, \quad
I_5 = 2.708 \, \text{А}.
\]

Все токи найдены с использованием I и II законов Кирхгофа.
