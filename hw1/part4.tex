Расчёт \textbf{баланса мощностей}, развиваемых источниками электрической эергии в цепи.

\subsection{Дано}
Параметры источников: \(J_6 = 1{,}95 \, \text{А}, \, E_5 = 7 \, \text{В}, \, E_2 = 34{,}5 \, \text{В} \) \\
Параметры резисторов: \(R_1 = 8 \, \Omega, \, R_2 = 6 \, \Omega, \, R_3 = 7 \, \Omega, \, R_4 = 2 \, \Omega, \, R_5 = 7 \, \Omega\) \\
Токи, полученные в результате вычислений частей 1-3:
\[
	I_1 = 2.116 \, \text{А}, \quad
	I_2 = 2.874 \, \text{А}, \quad
	I_3 = 0.758 \, \text{А}, \quad
	I_4 = -0.166 \, \text{А}, \quad
	I_5 = 2.708 \, \text{А}.
\]

\subsection{Найти}
$U_j$, мощности всех элементов цепи, суммарные мощности источников и приемников, показать, что соблюдается БМ

\subsection{Решение}
\begin{enumerate}
	\item Коэффициент затухания:

	      \[
		      \delta = \frac{R}{2L} = \frac{20}{2 \cdot 0,48} = 20,833 \, \text{с}^{-1}
	      \]

	\item Резонансная частота:

	      \[
		      \omega_c = \sqrt{\frac{1}{LC} - \delta^2} = \sqrt{\frac{1}{0,48 \cdot 300 \cdot 10^{-6}} - \frac{125^2}{6^2}} \approx 80,687 \, \text{с}^{-1}
	      \]

	\item Эксперементальное определение коэффициента затухания:

	      \[
		      \delta^* = \frac{\ln{\left(\frac{I_{m1}}{I_{m2}}\right)}}{T} = \frac{\ln{\left(\frac{0,071082}{0,031584}\right)}}{0,0781} = 10,386 \, \text{с}^{-1}
	      \]

	\item Эксперементальное определение резонансной частоты:

	    \[
	        \omega_c^* = \frac{2\pi}{T} = \frac{2\pi}{0,0781} = 80,451 \, \text{с}^{-1}
	    \]
\end{enumerate}


\subsection{Ответ}
$I_1 = 2.116 \, \text{А}, \, I_2 = 2.874 \, \text{А}, \, I_3 = 0.758 \, \text{А}, \, I_4 = -0.166 \, \text{А}, \, I_5 = 2.708 \, \text{А}, \, U_{J_6} = -11.624 \, \text{В},$

$P_{J_6} = 22.667 \, \text{Вт}, \, P_{E_2} = 99.123 \, \text{Вт}, \, P_{E_5} = 18.956 \, \text{Вт}, \, P_{R_1} = 35.820 \, \text{Вт}, \, P_{R_2} = 49.559 \, \text{Вт}, \, P_{R_3} = 4.022 \, \text{Вт}, \, P_{R_4} = 0.055 \, \text{Вт}, \, P_{R_5} = 51.333 \, \text{Вт},$

$P_{\Sigma \text{ист.}} = P_{\Sigma \text{потр.}} = 140.7 \, \text{Вт}$.
