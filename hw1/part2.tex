Расчёт значения всех неизвестных токов в представленной на рисунке 1 цепи с помощью \textbf{метода узловых напряжений (МУН)}.

\subsection{Найти}
Все неизвестные токи в цепи: \(I_1, I_2, I_3, I_4, I_5 = ?\) \\
(Используя только метод узловых напряжений)

\subsection{Решение}
\begin{enumerate}
	\item Расчёт действующего тока в цепи:
	      \[
		      \begin{gathered}
			      I = \frac{U}{Z} = \frac{U}{\sqrt{R^2 + X^2}} \\
			      X = -X_C, R = 0 \implies I = \frac{U}{X_C} = \frac{6}{111.96} = 0.0536 \, \text{А}
		      \end{gathered}
	      \]
	\item Расчёт фазового сдвига:
	      \[
		      \phi = \arctan\left(-\inf\right) = -90^{\circ}
	      \]
\end{enumerate}


\subsection{Ответ}
Рассчитанные значения неизвестных токов в цепи:

\[
	I_1 = 2.116 \, \text{А}, \quad
	I_2 = 2.874 \, \text{А}, \quad
	I_3 = 0.758 \, \text{А}, \quad
	I_4 = -0.166 \, \text{А}, \quad
	I_5 = 2.708 \, \text{А}.
\]

Все токи найдены с использованием Метода узловых напряжений и полностью совпадают со значениями, полученными с помощью I и II законов Кирхгофа, что подтверждает правильность расчётов \textit{первой} и \textit{второй} частей.
