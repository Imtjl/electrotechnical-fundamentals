\newcommand{\given}[1]{%
	\begin{figure}[H]
		\centering
		\begin{circuitikz}[american, scale=#1]

			\draw
			(0,0)
			to[R, l=$R_3$] (10,0)
			-- (10, -3)
			to[short, *-] (10, -4.1);

			\draw
			(10, -3)
			to[R, l=$R_5$] (7.5, -3)
			to[I, l=$E_5$] (5, -3)
			to[short, *-] (5, -3)
			to[R, l=$R_2$] (2.5, -3)
			to[I, l=$E_2$] (0, -3);

			\draw
			(0,0)
			-- (0,-3)
			to[short, *-] (0, -3)
			to[R, l=$R_1$] (0, -6)
			-- (5, -6) to[short, *-] (10, -6)
			-- (10, -4.9);

			\draw
			(5, -3)
			to[R, l=$R_4$] (5, -6);


			\draw
			(10,-4.5) circle(0.4cm)
			(10,-4.45) edge[thick, -{Straight Barb[length=2mm]}] (10, -4.44)
			(10,-4.35) edge[thick, -{Straight Barb[length=2mm]}] (10, -4.34)
			node at (10.75, -4.5) {$J_6$};

		\end{circuitikz}
		\caption{Исходная электрическая схема с обозначениями элементов}
	\end{figure}
}
