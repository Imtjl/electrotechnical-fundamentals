\documentclass[a4paper]{article}

\usepackage[14pt]{extsizes}
\usepackage[T2A]{fontenc}
\usepackage[russian]{babel}

\usepackage[left=20mm, top=15mm, right=15mm, bottom=20mm]{geometry}
\usepackage{listings}
\usepackage{xcolor}
\usepackage{tikz}
\usetikzlibrary{shapes.geometric, arrows.meta, positioning, calc, arrows, shapes.misc}
\usepackage{graphicx}
\usepackage{amsmath, amssymb} % For equations
\usepackage{booktabs} % For better tables
\usepackage{pgfplots} % For plotting graphs
\usepackage{caption} % For captioning tables and figures
\usepackage{float} % For precise float placement (images, tables)
\usepackage[hidelinks]{hyperref} % For table of contents to be clickable
\usepackage{bookmark}
\usepackage{multirow}
\usepackage{array}
\usepackage{cancel}
\usepackage{placeins}
\usepackage{enumitem}
\pgfplotsset{compat=1.17}
\usepackage{circuitikz}
\usetikzlibrary{decorations.markings} % For custom arrow positioning

% -----------------------------------------------------

% \input{../common/title-lab.tex}
\input{../common/title-hw.tex}
\input{../common/lstset.tex}
\input{../common/tikzset.tex}

\begin{document}

% -------------------------------
% Title page

% homework title
\hwtitle{1}{Расчёт цепей постоянного тока}{3331}{062}{Дворкин Борис Александрович}{14.10.2024}{04.12.2024}{}
\thispagestyle{empty}

% -------------------------------

% Enable text numbering
\newpage
\pagestyle{plain}
\setcounter{page}{1}

% -------------------------------

% autogenerated table of contents
\linespread{0.9}
\tableofcontents
\linespread{1}

% -------------------------------

\newpage
\section*{Цель работы}
\addcontentsline{toc}{section}{Цель работы}

Рассчитать значения всех неизвестных токов в цепи, используя:
\begin{enumerate}
	\item Законы \textbf{Кирхгофа}.
	\item Метод узловых напряжений (\textbf{МУН}).
\end{enumerate}
А также:
\begin{enumerate}[resume]
	\item Рассчитать ток любой ветви, содержащей источник ЭДС, методом эквивалентных преобразований (\textbf{МЭП}).
	\item Определить \textbf{напряжение}, приложенное к источнику тока. Определить \textbf{мощность} всех источников энергии, всех резистивных элементов, суммарную мощность источников цепи и суммарную мощность потребителей цепи.
\end{enumerate}

% -------------------------------

\section*{Дано (ч1-3)}
\addcontentsline{toc}{section}{Дано (ч1-3)}
\subsection*{Исходные параметры элементов цепи}
\addcontentsline{toc}{subsection}{Исходные параметры элементов цепи}
\stepcounter{subsection}

Параметры источников: \(J_6 = 1{,}95 \, \text{А}, \, E_5 = 7 \, \text{В}, \, E_2 = 34{,}5 \, \text{В} \) \\
Параметры резисторов: \(R_1 = 8 \, \Omega, \, R_2 = 6 \, \Omega, \, R_3 = 7 \, \Omega, \, R_4 = 2 \, \Omega, \, R_5 = 7 \, \Omega\)

\subsection*{Схема электрической цепи}
\addcontentsline{toc}{subsection}{Схема электрической цепи}
\stepcounter{subsection}

\newcommand{\given}[1]{%
	\begin{figure}[H]
		\centering
		\begin{circuitikz}[american, scale=#1]

			\draw
			(0,0)
			to[R, l=$R_3$] (10,0)
			-- (10, -3)
			to[short, *-] (10, -4.1);

			\draw
			(10, -3)
			to[R, l=$R_5$] (7.5, -3)
			to[I, l=$E_5$] (5, -3)
			to[short, *-] (5, -3)
			to[R, l=$R_2$] (2.5, -3)
			to[I, l=$E_2$] (0, -3);

			\draw
			(0,0)
			-- (0,-3)
			to[short, *-] (0, -3)
			to[R, l=$R_1$] (0, -6)
			-- (5, -6) to[short, *-] (10, -6)
			-- (10, -4.9);

			\draw
			(5, -3)
			to[R, l=$R_4$] (5, -6);


			\draw
			(10,-4.5) circle(0.4cm)
			(10,-4.45) edge[thick, -{Straight Barb[length=2mm]}] (10, -4.44)
			(10,-4.35) edge[thick, -{Straight Barb[length=2mm]}] (10, -4.34)
			node at (10.75, -4.5) {$J_6$};

		\end{circuitikz}
		\caption{Исходная электрическая схема с обозначениями элементов}
	\end{figure}
}



% -------------------------------

\newpage
\section*{Часть 1}
\addcontentsline{toc}{section}{Часть 1}
\stepcounter{section}
\subsection{Введение}
В данной части лабораторной работы произведены измерения действующих значений входного напряжения, тока и фазового сдвига между ними для девяти различных двухполюсников, а также произведены сравнения результатов с расчётными значениями.

\subsection{Параметры источника}

\subsection{Общие расчёты}
\begin{enumerate}
	\item Угловая частота:
	      \[
		      \omega = 2 \pi f = 2 \cdot 3.1416 \cdot 19.894 \approx 125 \, \text{рад/с} \\
	      \]

	\item Реактивная составляющая сопротивления ёмкостного элемента:
	      \[
		      X_c = \frac{1}{\omega C} = \frac{1}{125 \cdot 71.454 \cdot 10^{-6}} = 111.96 \, \text{Ом}
	      \]

	\item Реактивная составляющая сопротивления индуктивного элемента:
	      \[
		      X_L = \omega L = 125 \cdot 23.094 \cdot 10^{-3} = 2.887 \, \text{Ом}
	      \]

	\item Реактивная проводимость ёмкостного элемента:
	      \[
		      B_c = \omega C = 125 \cdot 71.454 \cdot 10^{-6} = 0.00893 \, \text{См}
	      \]

	\item Реактивная проводимость индуктивного элемента:
	      \[
		      B_k = \frac{X_L}{R_k^2 + X_L^2} = \frac{2.887}{5^2 + (2.887)^2} = 0.0866 \, \text{См}
	      \]
\end{enumerate}


\subsection{Двухполюсник 1}
\subsubsection{Схема исследуемой цепи}
dgfdgd
\subsubsection{Расчётные формулы и расчёты}
\subsubsection{Векторная диаграмма входного напряжения и тока}

\subsection{Двухполюсник 2}
\subsubsection{Схема исследуемой цепи}
\subsubsection{Расчётные формулы и расчёты}
\subsubsection{Векторная диаграмма входного напряжения и тока}

\subsection{Двухполюсник 3}
\subsubsection{Схема исследуемой цепи}
\subsubsection{Расчётные формулы и расчёты}
\subsubsection{Векторная диаграмма входного напряжения и тока}

\subsection{Двухполюсник 4}
\subsubsection{Схема исследуемой цепи}
\subsubsection{Расчётные формулы и расчёты}
\subsubsection{Векторная диаграмма входного напряжения и тока}

\subsection{Двухполюсник 5}
\subsubsection{Схема исследуемой цепи}
\subsubsection{Расчётные формулы и расчёты}
\subsubsection{Векторная диаграмма входного напряжения и тока}

\subsection{Двухполюсник 6}
\subsubsection{Схема исследуемой цепи}
\subsubsection{Расчётные формулы и расчёты}
\subsubsection{Векторная диаграмма входного напряжения и тока}

\subsection{Двухполюсник 7}
\subsubsection{Схема исследуемой цепи}
\subsubsection{Расчётные формулы и расчёты}
\subsubsection{Векторная диаграмма входного напряжения и тока}

\subsection{Двухполюсник 8}
\subsubsection{Схема исследуемой цепи}
\subsubsection{Расчётные формулы и расчёты}
\subsubsection{Векторная диаграмма входного напряжения и тока}

\subsection{Двухполюсник 9}
\subsubsection{Схема исследуемой цепи}
\subsubsection{Расчётные формулы и расчёты}
\subsubsection{Векторная диаграмма входного напряжения и тока}

\subsection{Заполненная таблица 2.2}

\subsection{Выводы}


% -------------------------------

\newpage
\section*{Часть 2}
\addcontentsline{toc}{section}{Часть 2}
\stepcounter{section}
\subsection{Апериодический процесс}

\subsubsection{Схема исследуемой цепи}
На рисунке 1.1 представлена схема замещения источника электрической энергии постоянного тока и нагрузки, созданная в приложении LTspice.

% \begin{figure}[H]
% 	\centering
% 	\includegraphics[width=0.6\textwidth]{rcl-schema.png} % Make sure the path to the image is correct
% 	\caption{Схема замещения источника электрической энергии в LTspice.}
% \end{figure}

\subsubsection{Расчётные формулы и расчёты}

\subsubsection{Графики переходных процессов}

\subsubsection{Таблица результатов 4.4}

\subsection{Колебательный процесс}

\subsubsection{Схема исследуемой цепи}
На рисунке 1.1 представлена схема замещения источника электрической энергии постоянного тока и нагрузки, созданная в приложении LTspice.

% \begin{figure}[H]
% 	\centering
% 	\includegraphics[width=0.6\textwidth]{rcl-schema.png} % Make sure the path to the image is correct
% 	\caption{Схема замещения источника электрической энергии в LTspice.}
% \end{figure}

\subsubsection{Расчётные формулы и расчёты}

\subsubsection{Графики переходных процессов}

\subsubsection{Таблица результатов 4.5}

\subsection{Выводы по второй части}


% -------------------------------

\newpage
\section*{Часть 3}
\addcontentsline{toc}{section}{Часть 3}
\stepcounter{section}
Расчёт значения тока через источник $E_2$ в представленной на рисунке 1 цепи с помощью \textbf{метода эквивалентных преобразований (МЭП)}.

\subsection{Найти}
Ток через $E_2$: \(I_2 = ?\) \\
(Используя только метод эквивалентных преобразований)

\subsection{Решение}
\begin{enumerate}
	\item Применим типивые эквивалентные преобразования к \textbf{исходной схеме}:
	      \newcommand{\given}[1]{%
	\begin{figure}[H]
		\centering
		\begin{circuitikz}[american, scale=#1]

			\draw
			(0,0)
			to[R, l=$R_3$] (10,0)
			-- (10, -3)
			to[short, *-] (10, -4.1);

			\draw
			(10, -3)
			to[R, l=$R_5$] (7.5, -3)
			to[I, l=$E_5$] (5, -3)
			to[short, *-] (5, -3)
			to[R, l=$R_2$] (2.5, -3)
			to[I, l=$E_2$] (0, -3);

			\draw
			(0,0)
			-- (0,-3)
			to[short, *-] (0, -3)
			to[R, l=$R_1$] (0, -6)
			-- (5, -6) to[short, *-] (10, -6)
			-- (10, -4.9);

			\draw
			(5, -3)
			to[R, l=$R_4$] (5, -6);


			\draw
			(10,-4.5) circle(0.4cm)
			(10,-4.45) edge[thick, -{Straight Barb[length=2mm]}] (10, -4.44)
			(10,-4.35) edge[thick, -{Straight Barb[length=2mm]}] (10, -4.34)
			node at (10.75, -4.5) {$J_6$};

		\end{circuitikz}
		\caption{Исходная электрическая схема с обозначениями элементов}
	\end{figure}
}

	      \given{1}
	\item Расщепляем $J_6$ на $R_4, R_5, E_5$:
	      \begin{figure}[H]
	\centering
	\begin{circuitikz}[american, scale=1]

		\draw
		(0,0)
		to[R, l=$R_3$] (10,0)
		-- (10, -2)
		to[short, *-] (10, -2);

		\draw
		(10, -2)
		to[R, l_=$R_5$] (7, -2)
		to[I, l=$E_5$] (5, -2)
		to[short, *-] (5, -2)
		to[R, l=$R_2$] (2.5, -2)
		to[I, l=$E_2$] (0, -2);

		\draw
		(0,0)
		-- (0,-2)
		to[short, *-] (0, -3)
		to[R, l=$R_1$] (0, -6)
		-- (5, -6) to[short, *-] (5, -6);

		\draw
		(5, -2)
		-- (5, -3)
		to[R, l_=$R_4$] (5, -6);


		\draw
		(9.5, -2)
		to[short, *-] (9.5, -2)
		-- (9.5, -3)
		-- (8.85, -3);

		\draw
		(7.5, -2)
		to[short, *-] (7.5, -2)
		-- (7.5, -3)
		-- (8.05, -3);

		\draw[rotate around={-90:(8.5,-3)}]
		(8.5,-3.05) circle(0.4cm)
		(8.5,-2.95) edge[thick, -{Straight Barb[length=2mm]}] (8.5, -2.94)
		(8.5,-2.85) edge[thick, -{Straight Barb[length=2mm]}] (8.5, -2.84)
		node at (9.25, -3) {$J_{\text{э2}}$};

		\draw
		(5, -3.5)
		to[short, *-] (5, -3.5)
		-- (6, -3.5)
		-- (6, -4.1);

		\draw
		(5, -5.5)
		to[short, *-] (5, -5.5)
		-- (6, -5.5)
		-- (6, -4.9);

		\draw
		(6,-4.5) circle(0.4cm)
		(6,-4.45) edge[thick, -{Straight Barb[length=2mm]}] (6, -4.44)
		(6,-4.35) edge[thick, -{Straight Barb[length=2mm]}] (6, -4.34)
		node at (6.75, -4.5) {$J_{\text{э1}}$};


	\end{circuitikz}
\end{figure}

	\item $J_{\text{э}1}$ \parallel $R_4 \rightarrow E_4$, $J_{\text{э}2}$ \parallel $R_5 \rightarrow E_{5'}$:
	      \begin{figure}[H]
	\centering
	\begin{circuitikz}[american, scale=1]

		\draw
		(0,0)
		to[R, l=$R_3$] (10,0)
		-- (10, -2)
		to[short, *-] (10, -2);

		\draw
		(10, -2)
		to[R, l=$R_5$] (8, -2);

		\draw
		(7, -2) to[I, l_=$E_{5'}$] (8, -2);

		\draw
		(7, -2)
		to[I, l=$E_5$] (5, -2)
		to[short, *-] (5, -2)
		to[R, l=$R_2$] (2.5, -2)
		to[I, l=$E_2$] (0, -2);

		\draw
		(0,0)
		-- (0,-2)
		to[short, *-] (0, -5)
		to[R, l=$R_{14}$] (3, -5)
		to[I, l=$E_4$] (5, -5)
		-- (5, -2);
	\end{circuitikz}
\end{figure}


	      \[
		      \begin{gathered}
			      E_4 = R_4 \cdot J_{\text{э1}} = 2 \cdot 1.95 = 3.9 \, \text{В} \\
			      E_{5'} = R_5 \cdot J_{\text{э2}} = 7 \cdot 1.95 = 13.65 \, \text{В} \\
			      R_{35} = R_3 + R_5 = 14 \, \Omega \\
			      E_{55'} = E_5 - E_{5'} = 6.65 \, \text{В}
		      \end{gathered}
	      \]
	\item $R_{14}, E_4$ \parallel $R_{35}, E_{55'}$:
	      \begin{figure}[H]
	\centering
	\begin{circuitikz}[american, scale=1]

		\draw
		(0,0)
		-- (10, 0)
		-- (10, -2)
		to[R, l=$R_{1435}$] (8, -2)
		to[I, l=$E_{455'}$] (5, -2)
		to[R, l=$R_2$] (2.5, -2)
		to[I, l=$E_2$] (0, -2)
		-- (0, 0);

	\end{circuitikz}
\end{figure}


	      \[
		      \begin{gathered}
			      R_{1435} = \frac{1}{\frac{1}{R_{14}} + \frac{1}{R_{35}}} = \frac{1}{\frac{1}{10} + \frac{1}{14}} = 5.833 \, \Omega \\
			      E_{455'} = R_{1435} \cdot \left(\frac{E_4}{R_{14}} - \frac{E_{55'}}{R_{35}}\right) = 5.833 \cdot \left(\frac{3.9}{10} - \frac{6.65}{14}\right) = -0.496 \, \text{В} \\
		      \end{gathered}
	      \]
	\item Схема сведена к одноконтурной относительно ветви с искомым током. Искомый ток $I2$ определим с использованием ЗКII:
	      \[
		      \begin{gathered}
			      I_2 \cdot (R_2 + R_{1345}) = E_2 + E_{455'} \Leftrightarrow I_2 = \frac{E_2 + E_{455'}}{R_2 + R_{1345}} \\
			      I_2 = \frac{34.5 - 0.496}{6+5.833} = 2.874 \, \text{А}
		      \end{gathered}
	      \]

\end{enumerate}


\subsection{Ответ}
Рассчитанное значение неизвестного тока через $E_2$:

\[
	I_2 = 2.874 \, \text{А}.
\]

Ток найден с помощью метода эквивалентных преобразований и совпадает со значением, найденным с помощью Законов Кирхгофа и метода узловых напряжений, что подтверждает правильность расчётов \textit{первой}, \textit{второй} и \textit{третьей} частей.


% -------------------------------

\section*{Часть 4}
\addcontentsline{toc}{section}{Часть 4}
\stepcounter{section}
Расчёт \textbf{баланса мощностей}, развиваемых источниками электрической эергии в цепи.

\subsection{Дано}
Параметры источников: \(J_6 = 1{,}95 \, \text{А}, \, E_5 = 7 \, \text{В}, \, E_2 = 34{,}5 \, \text{В} \) \\
Параметры резисторов: \(R_1 = 8 \, \Omega, \, R_2 = 6 \, \Omega, \, R_3 = 7 \, \Omega, \, R_4 = 2 \, \Omega, \, R_5 = 7 \, \Omega\) \\
Токи, полученные в результате вычислений частей 1-3:
\[
	I_1 = 2.116 \, \text{А}, \quad
	I_2 = 2.874 \, \text{А}, \quad
	I_3 = 0.758 \, \text{А}, \quad
	I_4 = -0.166 \, \text{А}, \quad
	I_5 = 2.708 \, \text{А}.
\]

\subsection{Найти}
$U_j$, мощности всех элементов цепи, суммарные мощности источников и приемников, показать, что соблюдается БМ

\subsection{Решение}
\begin{enumerate}
	\item Коэффициент затухания:

	      \[
		      \delta = \frac{R}{2L} = \frac{20}{2 \cdot 0,48} = 20,833 \, \text{с}^{-1}
	      \]

	\item Резонансная частота:

	      \[
		      \omega_c = \sqrt{\frac{1}{LC} - \delta^2} = \sqrt{\frac{1}{0,48 \cdot 300 \cdot 10^{-6}} - \frac{125^2}{6^2}} \approx 80,687 \, \text{с}^{-1}
	      \]

	\item Эксперементальное определение коэффициента затухания:

	      \[
		      \delta^* = \frac{\ln{\left(\frac{I_{m1}}{I_{m2}}\right)}}{T} = \frac{\ln{\left(\frac{0,071082}{0,031584}\right)}}{0,0781} = 10,386 \, \text{с}^{-1}
	      \]

	\item Эксперементальное определение резонансной частоты:

	    \[
	        \omega_c^* = \frac{2\pi}{T} = \frac{2\pi}{0,0781} = 80,451 \, \text{с}^{-1}
	    \]
\end{enumerate}


\subsection{Ответ}
$I_1 = 2.116 \, \text{А}, \, I_2 = 2.874 \, \text{А}, \, I_3 = 0.758 \, \text{А}, \, I_4 = -0.166 \, \text{А}, \, I_5 = 2.708 \, \text{А}, \, U_{J_6} = -11.624 \, \text{В},$

$P_{J_6} = 22.667 \, \text{Вт}, \, P_{E_2} = 99.123 \, \text{Вт}, \, P_{E_5} = 18.956 \, \text{Вт}, \, P_{R_1} = 35.820 \, \text{Вт}, \, P_{R_2} = 49.559 \, \text{Вт}, \, P_{R_3} = 4.022 \, \text{Вт}, \, P_{R_4} = 0.055 \, \text{Вт}, \, P_{R_5} = 51.333 \, \text{Вт},$

$P_{\Sigma \text{ист.}} = P_{\Sigma \text{потр.}} = 140.7 \, \text{Вт}$.


% -------------------------------

\end{document}
