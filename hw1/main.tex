\documentclass[a4paper]{article}

\usepackage[14pt]{extsizes}
\usepackage[T2A]{fontenc}
\usepackage[russian]{babel}

\usepackage[left=20mm, top=15mm, right=15mm, bottom=20mm]{geometry}
\usepackage{listings}
\usepackage{xcolor}
\usepackage{tikz}
\usetikzlibrary{shapes.geometric, arrows.meta, positioning, calc, arrows, shapes.misc}
\usepackage{graphicx}
\usepackage{amsmath, amssymb} % For equations
\usepackage{booktabs} % For better tables
\usepackage{pgfplots} % For plotting graphs
\usepackage{caption} % For captioning tables and figures
\usepackage{float} % For precise float placement (images, tables)
\usepackage[hidelinks]{hyperref} % For table of contents to be clickable
\usepackage{bookmark}
\usepackage{multirow}
\usepackage{array}
\usepackage{cancel}
\usepackage{placeins}
\usepackage{enumitem}
\pgfplotsset{compat=1.17}
\usepackage{circuitikz}
\usetikzlibrary{decorations.markings} % For custom arrow positioning

% -----------------------------------------------------

% \input{../common/title-lab.tex}
\input{../common/title-hw.tex}
\input{../common/lstset.tex}
\input{../common/tikzset.tex}

\begin{document}

% -------------------------------
% Title page

% homework title
\hwtitle{1}{Расчёт цепей постоянного тока}{3331}{062}{Дворкин Борис Александрович}{14.10.2024}{04.12.2024}{}
\thispagestyle{empty}

% -------------------------------

% Enable text numbering
\newpage
\pagestyle{plain}
\setcounter{page}{1}

% -------------------------------

% autogenerated table of contents
\linespread{0.9}
\tableofcontents
\linespread{1}

% -------------------------------

\newpage
\section*{Цель работы}
\addcontentsline{toc}{section}{Цель работы}

Рассчитать значения всех неизвестных токов в цепи, используя:
\begin{enumerate}
	\item Законы \textbf{Кирхгофа}.
	\item Метод узловых напряжений (\textbf{МУН}).
\end{enumerate}
А также:
\begin{enumerate}[resume]
	\item Рассчитать ток любой ветви, содержащей источник ЭДС, методом эквивалентного генератора (\textbf{МЭГ}).
	\item Определить \textbf{напряжение}, приложенное к источнику тока. Определить \textbf{мощность} всех источников энергии, всех резистивных элементов, суммарную мощность источников цепи и суммарную мощность потребителей цепи.
\end{enumerate}

% -------------------------------

\section*{Дано}
\addcontentsline{toc}{section}{Дано}
\subsection*{Исходные параметры элементов цепи}
\addcontentsline{toc}{subsection}{Исходные параметры элементов цепи}
\stepcounter{subsection}

Параметры источников: \(J_6 = 1{,}95 \, \text{А}, \, E_5 = 7 \, \text{В}, \, E_2 = 34{,}5 \, \text{В} \) \\
Параметры резисторов: \(R_1 = 8 \, \Omega, \, R_2 = 6 \, \Omega, \, R_3 = 7 \, \Omega, \, R_4 = 2 \, \Omega, \, R_5 = 7 \, \Omega\)

\subsection*{Схема электрической цепи}
\addcontentsline{toc}{subsection}{Схема электрической цепи}
\stepcounter{subsection}

\newcommand{\given}[1]{%
	\begin{figure}[H]
		\centering
		\begin{circuitikz}[american, scale=#1]

			\draw
			(0,0)
			to[R, l=$R_3$] (10,0)
			-- (10, -3)
			to[short, *-] (10, -4.1);

			\draw
			(10, -3)
			to[R, l=$R_5$] (7.5, -3)
			to[I, l=$E_5$] (5, -3)
			to[short, *-] (5, -3)
			to[R, l=$R_2$] (2.5, -3)
			to[I, l=$E_2$] (0, -3);

			\draw
			(0,0)
			-- (0,-3)
			to[short, *-] (0, -3)
			to[R, l=$R_1$] (0, -6)
			-- (5, -6) to[short, *-] (10, -6)
			-- (10, -4.9);

			\draw
			(5, -3)
			to[R, l=$R_4$] (5, -6);


			\draw
			(10,-4.5) circle(0.4cm)
			(10,-4.45) edge[thick, -{Straight Barb[length=2mm]}] (10, -4.44)
			(10,-4.35) edge[thick, -{Straight Barb[length=2mm]}] (10, -4.34)
			node at (10.75, -4.5) {$J_6$};

		\end{circuitikz}
		\caption{Исходная электрическая схема с обозначениями элементов}
	\end{figure}
}



% -------------------------------

\newpage
\section*{Часть 1}
\addcontentsline{toc}{section}{Часть 1}
\stepcounter{section}
\subsection{Введение}
В данной части лабораторной работы произведены измерения действующих значений входного напряжения, тока и фазового сдвига между ними для девяти различных двухполюсников, а также произведены сравнения результатов с расчётными значениями.

\subsection{Параметры источника}

\subsection{Общие расчёты}
\begin{enumerate}
	\item Угловая частота:
	      \[
		      \omega = 2 \pi f = 2 \cdot 3.1416 \cdot 19.894 \approx 125 \, \text{рад/с} \\
	      \]

	\item Реактивная составляющая сопротивления ёмкостного элемента:
	      \[
		      X_c = \frac{1}{\omega C} = \frac{1}{125 \cdot 71.454 \cdot 10^{-6}} = 111.96 \, \text{Ом}
	      \]

	\item Реактивная составляющая сопротивления индуктивного элемента:
	      \[
		      X_L = \omega L = 125 \cdot 23.094 \cdot 10^{-3} = 2.887 \, \text{Ом}
	      \]

	\item Реактивная проводимость ёмкостного элемента:
	      \[
		      B_c = \omega C = 125 \cdot 71.454 \cdot 10^{-6} = 0.00893 \, \text{См}
	      \]

	\item Реактивная проводимость индуктивного элемента:
	      \[
		      B_k = \frac{X_L}{R_k^2 + X_L^2} = \frac{2.887}{5^2 + (2.887)^2} = 0.0866 \, \text{См}
	      \]
\end{enumerate}


\subsection{Двухполюсник 1}
\subsubsection{Схема исследуемой цепи}
dgfdgd
\subsubsection{Расчётные формулы и расчёты}
\subsubsection{Векторная диаграмма входного напряжения и тока}

\subsection{Двухполюсник 2}
\subsubsection{Схема исследуемой цепи}
\subsubsection{Расчётные формулы и расчёты}
\subsubsection{Векторная диаграмма входного напряжения и тока}

\subsection{Двухполюсник 3}
\subsubsection{Схема исследуемой цепи}
\subsubsection{Расчётные формулы и расчёты}
\subsubsection{Векторная диаграмма входного напряжения и тока}

\subsection{Двухполюсник 4}
\subsubsection{Схема исследуемой цепи}
\subsubsection{Расчётные формулы и расчёты}
\subsubsection{Векторная диаграмма входного напряжения и тока}

\subsection{Двухполюсник 5}
\subsubsection{Схема исследуемой цепи}
\subsubsection{Расчётные формулы и расчёты}
\subsubsection{Векторная диаграмма входного напряжения и тока}

\subsection{Двухполюсник 6}
\subsubsection{Схема исследуемой цепи}
\subsubsection{Расчётные формулы и расчёты}
\subsubsection{Векторная диаграмма входного напряжения и тока}

\subsection{Двухполюсник 7}
\subsubsection{Схема исследуемой цепи}
\subsubsection{Расчётные формулы и расчёты}
\subsubsection{Векторная диаграмма входного напряжения и тока}

\subsection{Двухполюсник 8}
\subsubsection{Схема исследуемой цепи}
\subsubsection{Расчётные формулы и расчёты}
\subsubsection{Векторная диаграмма входного напряжения и тока}

\subsection{Двухполюсник 9}
\subsubsection{Схема исследуемой цепи}
\subsubsection{Расчётные формулы и расчёты}
\subsubsection{Векторная диаграмма входного напряжения и тока}

\subsection{Заполненная таблица 2.2}

\subsection{Выводы}


% -------------------------------

\newpage
\section*{Часть 2}
\addcontentsline{toc}{section}{Часть 2}
\stepcounter{section}
% \subsection{Введение}
В данной части лабораторной работы произведены измерения действующих значений входного напряжения, тока и фазового сдвига между ними для девяти различных двухполюсников, а также произведены сравнения результатов с расчётными значениями.

\subsection{Параметры источника}

\subsection{Общие расчёты}
\begin{enumerate}
	\item Угловая частота:
	      \[
		      \omega = 2 \pi f = 2 \cdot 3.1416 \cdot 19.894 \approx 125 \, \text{рад/с} \\
	      \]

	\item Реактивная составляющая сопротивления ёмкостного элемента:
	      \[
		      X_c = \frac{1}{\omega C} = \frac{1}{125 \cdot 71.454 \cdot 10^{-6}} = 111.96 \, \text{Ом}
	      \]

	\item Реактивная составляющая сопротивления индуктивного элемента:
	      \[
		      X_L = \omega L = 125 \cdot 23.094 \cdot 10^{-3} = 2.887 \, \text{Ом}
	      \]

	\item Реактивная проводимость ёмкостного элемента:
	      \[
		      B_c = \omega C = 125 \cdot 71.454 \cdot 10^{-6} = 0.00893 \, \text{См}
	      \]

	\item Реактивная проводимость индуктивного элемента:
	      \[
		      B_k = \frac{X_L}{R_k^2 + X_L^2} = \frac{2.887}{5^2 + (2.887)^2} = 0.0866 \, \text{См}
	      \]
\end{enumerate}


\subsection{Двухполюсник 1}
\subsubsection{Схема исследуемой цепи}
dgfdgd
\subsubsection{Расчётные формулы и расчёты}
\subsubsection{Векторная диаграмма входного напряжения и тока}

\subsection{Двухполюсник 2}
\subsubsection{Схема исследуемой цепи}
\subsubsection{Расчётные формулы и расчёты}
\subsubsection{Векторная диаграмма входного напряжения и тока}

\subsection{Двухполюсник 3}
\subsubsection{Схема исследуемой цепи}
\subsubsection{Расчётные формулы и расчёты}
\subsubsection{Векторная диаграмма входного напряжения и тока}

\subsection{Двухполюсник 4}
\subsubsection{Схема исследуемой цепи}
\subsubsection{Расчётные формулы и расчёты}
\subsubsection{Векторная диаграмма входного напряжения и тока}

\subsection{Двухполюсник 5}
\subsubsection{Схема исследуемой цепи}
\subsubsection{Расчётные формулы и расчёты}
\subsubsection{Векторная диаграмма входного напряжения и тока}

\subsection{Двухполюсник 6}
\subsubsection{Схема исследуемой цепи}
\subsubsection{Расчётные формулы и расчёты}
\subsubsection{Векторная диаграмма входного напряжения и тока}

\subsection{Двухполюсник 7}
\subsubsection{Схема исследуемой цепи}
\subsubsection{Расчётные формулы и расчёты}
\subsubsection{Векторная диаграмма входного напряжения и тока}

\subsection{Двухполюсник 8}
\subsubsection{Схема исследуемой цепи}
\subsubsection{Расчётные формулы и расчёты}
\subsubsection{Векторная диаграмма входного напряжения и тока}

\subsection{Двухполюсник 9}
\subsubsection{Схема исследуемой цепи}
\subsubsection{Расчётные формулы и расчёты}
\subsubsection{Векторная диаграмма входного напряжения и тока}

\subsection{Заполненная таблица 2.2}

\subsection{Выводы}


% -------------------------------

\section*{Часть 3}
\addcontentsline{toc}{section}{Часть 3}
\stepcounter{section}
% \subsection{Введение}
В данной части лабораторной работы произведены измерения действующих значений входного напряжения, тока и фазового сдвига между ними для девяти различных двухполюсников, а также произведены сравнения результатов с расчётными значениями.

\subsection{Параметры источника}

\subsection{Общие расчёты}
\begin{enumerate}
	\item Угловая частота:
	      \[
		      \omega = 2 \pi f = 2 \cdot 3.1416 \cdot 19.894 \approx 125 \, \text{рад/с} \\
	      \]

	\item Реактивная составляющая сопротивления ёмкостного элемента:
	      \[
		      X_c = \frac{1}{\omega C} = \frac{1}{125 \cdot 71.454 \cdot 10^{-6}} = 111.96 \, \text{Ом}
	      \]

	\item Реактивная составляющая сопротивления индуктивного элемента:
	      \[
		      X_L = \omega L = 125 \cdot 23.094 \cdot 10^{-3} = 2.887 \, \text{Ом}
	      \]

	\item Реактивная проводимость ёмкостного элемента:
	      \[
		      B_c = \omega C = 125 \cdot 71.454 \cdot 10^{-6} = 0.00893 \, \text{См}
	      \]

	\item Реактивная проводимость индуктивного элемента:
	      \[
		      B_k = \frac{X_L}{R_k^2 + X_L^2} = \frac{2.887}{5^2 + (2.887)^2} = 0.0866 \, \text{См}
	      \]
\end{enumerate}


\subsection{Двухполюсник 1}
\subsubsection{Схема исследуемой цепи}
dgfdgd
\subsubsection{Расчётные формулы и расчёты}
\subsubsection{Векторная диаграмма входного напряжения и тока}

\subsection{Двухполюсник 2}
\subsubsection{Схема исследуемой цепи}
\subsubsection{Расчётные формулы и расчёты}
\subsubsection{Векторная диаграмма входного напряжения и тока}

\subsection{Двухполюсник 3}
\subsubsection{Схема исследуемой цепи}
\subsubsection{Расчётные формулы и расчёты}
\subsubsection{Векторная диаграмма входного напряжения и тока}

\subsection{Двухполюсник 4}
\subsubsection{Схема исследуемой цепи}
\subsubsection{Расчётные формулы и расчёты}
\subsubsection{Векторная диаграмма входного напряжения и тока}

\subsection{Двухполюсник 5}
\subsubsection{Схема исследуемой цепи}
\subsubsection{Расчётные формулы и расчёты}
\subsubsection{Векторная диаграмма входного напряжения и тока}

\subsection{Двухполюсник 6}
\subsubsection{Схема исследуемой цепи}
\subsubsection{Расчётные формулы и расчёты}
\subsubsection{Векторная диаграмма входного напряжения и тока}

\subsection{Двухполюсник 7}
\subsubsection{Схема исследуемой цепи}
\subsubsection{Расчётные формулы и расчёты}
\subsubsection{Векторная диаграмма входного напряжения и тока}

\subsection{Двухполюсник 8}
\subsubsection{Схема исследуемой цепи}
\subsubsection{Расчётные формулы и расчёты}
\subsubsection{Векторная диаграмма входного напряжения и тока}

\subsection{Двухполюсник 9}
\subsubsection{Схема исследуемой цепи}
\subsubsection{Расчётные формулы и расчёты}
\subsubsection{Векторная диаграмма входного напряжения и тока}

\subsection{Заполненная таблица 2.2}

\subsection{Выводы}


% -------------------------------

\section*{Часть 4}
\addcontentsline{toc}{section}{Часть 4}
\stepcounter{section}
% \subsection{Введение}
В данной части лабораторной работы произведены измерения действующих значений входного напряжения, тока и фазового сдвига между ними для девяти различных двухполюсников, а также произведены сравнения результатов с расчётными значениями.

\subsection{Параметры источника}

\subsection{Общие расчёты}
\begin{enumerate}
	\item Угловая частота:
	      \[
		      \omega = 2 \pi f = 2 \cdot 3.1416 \cdot 19.894 \approx 125 \, \text{рад/с} \\
	      \]

	\item Реактивная составляющая сопротивления ёмкостного элемента:
	      \[
		      X_c = \frac{1}{\omega C} = \frac{1}{125 \cdot 71.454 \cdot 10^{-6}} = 111.96 \, \text{Ом}
	      \]

	\item Реактивная составляющая сопротивления индуктивного элемента:
	      \[
		      X_L = \omega L = 125 \cdot 23.094 \cdot 10^{-3} = 2.887 \, \text{Ом}
	      \]

	\item Реактивная проводимость ёмкостного элемента:
	      \[
		      B_c = \omega C = 125 \cdot 71.454 \cdot 10^{-6} = 0.00893 \, \text{См}
	      \]

	\item Реактивная проводимость индуктивного элемента:
	      \[
		      B_k = \frac{X_L}{R_k^2 + X_L^2} = \frac{2.887}{5^2 + (2.887)^2} = 0.0866 \, \text{См}
	      \]
\end{enumerate}


\subsection{Двухполюсник 1}
\subsubsection{Схема исследуемой цепи}
dgfdgd
\subsubsection{Расчётные формулы и расчёты}
\subsubsection{Векторная диаграмма входного напряжения и тока}

\subsection{Двухполюсник 2}
\subsubsection{Схема исследуемой цепи}
\subsubsection{Расчётные формулы и расчёты}
\subsubsection{Векторная диаграмма входного напряжения и тока}

\subsection{Двухполюсник 3}
\subsubsection{Схема исследуемой цепи}
\subsubsection{Расчётные формулы и расчёты}
\subsubsection{Векторная диаграмма входного напряжения и тока}

\subsection{Двухполюсник 4}
\subsubsection{Схема исследуемой цепи}
\subsubsection{Расчётные формулы и расчёты}
\subsubsection{Векторная диаграмма входного напряжения и тока}

\subsection{Двухполюсник 5}
\subsubsection{Схема исследуемой цепи}
\subsubsection{Расчётные формулы и расчёты}
\subsubsection{Векторная диаграмма входного напряжения и тока}

\subsection{Двухполюсник 6}
\subsubsection{Схема исследуемой цепи}
\subsubsection{Расчётные формулы и расчёты}
\subsubsection{Векторная диаграмма входного напряжения и тока}

\subsection{Двухполюсник 7}
\subsubsection{Схема исследуемой цепи}
\subsubsection{Расчётные формулы и расчёты}
\subsubsection{Векторная диаграмма входного напряжения и тока}

\subsection{Двухполюсник 8}
\subsubsection{Схема исследуемой цепи}
\subsubsection{Расчётные формулы и расчёты}
\subsubsection{Векторная диаграмма входного напряжения и тока}

\subsection{Двухполюсник 9}
\subsubsection{Схема исследуемой цепи}
\subsubsection{Расчётные формулы и расчёты}
\subsubsection{Векторная диаграмма входного напряжения и тока}

\subsection{Заполненная таблица 2.2}

\subsection{Выводы}


% -------------------------------

% Other main content
% \input{chart}
% \input{code}
% \input{examples}
% \input{conclusion}

\end{document}
